\documentclass{techreport}

%\nochangebars

\newcommand{\Omit}[1]{}

\sloppy
\begin{document}
\title{21-651: General Topology}
\author{Di Wang (diw3)}
\maketitle

\section{Sets and Ordering}

\begin{definition}
	$2$ sets $A,B$ are said to be \emph{equipotent}, written $A \sim B$, if there exists a bijection between them.
\end{definition}

\begin{remark}
	Equipotence is an equivalence relation.
\end{remark}

\begin{definition}\
	\begin{itemize}
		\item A set is said to be \emph{finite} if it is equipotent to a set $\{1,\cdots,n\} \subset \bbN$ for some $n \in \bbN$.
		\item A set is said to be \emph{countable} if it is equipotent to some subset of $\bbN$.
		\item A set is said to be \emph{uncountable} if it is not countable.
	\end{itemize}
\end{definition}

\begin{remark}
	$\bbR$ is uncountable.
\end{remark}
\begin{proof}
	Suppose $\bbR$ is countable.
	Then there exists $S \subset \bbN^+$ such that $\bbR \sim S$.
	In other words, there exists a bijection $f : \bbR \to S$.
	Define $g$ to be the restriction of $f$ on $[0,1]$.
	Then $g : [0,1] \to f([0,1])$ is also a bijection.
	Define $S' \coloneqq f([0,1]) \subset S \subset \bbN^+$.
	Define a sequence $\{x_n\}_{n \in \bbN^+} \subset \{0,1\}$ in the following way:
	\begin{itemize}
		\item If $n \in S'$, let $r \coloneqq g^{-1}(n)$.
		Then there exists a sequence $\{ r_n\}_{n \in \bbN^+}$ such that $r = \sum_{n=1}^\infty r_n \cdot 2^{-n}$.
		Define $x_n \coloneqq 1 - r_n$.
		\item If $n \not\in S'$, define $x_n \coloneqq 0$.
	\end{itemize}
	Define $x \coloneqq \sum_{n=1}^\infty x_n \cdot 2^{-n} \in [0,1]$.
	We claim that $g(x) \not\in S'$.
	If $g(x) \in S'$, then $x = g^{-1}(g(x))$. By the definition of $x$ we know that $x_{g(x)} = 1-x_{g(x)}$, which leads to a contradiction.
	Hence $g(x) \not\in S'$, and therefore $g$ is not a bijection.
	Then we can conclude that $\bbR$ is not countable, i.e., indeed uncountable.
\end{proof}

\begin{theorem}[Cantor-Bernstein-Schr{\"o}der]
	If there exist injections $f:A \to B$ and $g : B \to A$, then $A \sim B$.
\end{theorem}

\begin{definition}
	Let $X$ be a set.
	The \emph{power set} of $X$, written $\wp(X)$, is the set of all subsets of $X$.
\end{definition}

\begin{lemma}[Cantor]
	Let $X$ be a set. There is no surjection from $X$ to $\wp(X)$.
\end{lemma}

\begin{remark}
	Let $X$ be a set. Then
	\[
	\wp(X) \sim \{0,1\}^X \coloneqq \{ f : X \to \{0,1\} \}.
	\]
\end{remark}

\begin{definition}
	A \emph{partial order} on a set $X$ is a binary relation on $X$, written ${\le}$, if
	\begin{itemize}
		\item (reflexivity): $x \le x$ for all $x \in X$,
		\item (antisymmetry): $x \le y$ and $y \le x$ imply $x = y$ for all $x,y \in X$, and
		\item (transitivity): $x \le y$ and $y \le z$ imply $x \le z$ for all $x,y,z \in X$.
	\end{itemize}
\end{definition}

\begin{example}
	Let $X \coloneqq \wp(\bbR)$.
	Then $A \le B \coloneqq A \subset B$ is a partial order on $X$.
\end{example}

\begin{definition}
	A partial ordering $(X,{\le})$ is a \emph{linear ordering} if for all $x,y \in X$, we have either $x \le y$ or $y \le x$.
\end{definition}

\begin{definition}
	Let $(X,{\le})$ be a partial ordering.
	Let $E \subset X$.
	\begin{itemize}
		\item $x \in X$ is said to be an \emph{upper bound} of $E$ if for all $y \in E$, we have $y \le x$.
		\item $x \in E$ is said to be a \emph{maximal element} of $E$ if there does not exist $y \in E$ such that $x < y$.
		\item If $x \in E$ and for all $y \in E$ we have $y \le x$, then $x$ is the \emph{greatest element} of $E$.
	\end{itemize}
	\emph{Lower bounds}, \emph{minimal elements}, and \emph{least elements} are defined in a similar way.
\end{definition}

\begin{definition}
	A poset $(X,{\le})$ is said to be \emph{well-ordered} if it is linearly ordered and every nonempty subset of $X$ has the least element.
\end{definition}

\begin{theorem}
	The following three are equivalent:
	\begin{itemize}
		\item \emph{Axiom of Choice:} Given a collection $\calA$ of nonempty disjoint sets, there exists a set $C$ such that (i) $C \subset \bigcup \calA$, and (ii) for all $A \in \calA$, $C \cap A$ has exactly one element.
		\item \emph{Zorn's Lemma:} Suppose $(X,{\le})$ is a poset. If every chain in $X$ has an upper bound, then $X$ has a maximal element.
		\item \emph{Well Ordering Lemma (Zermelo's Theorem):} Every nonempty set $X$ admits a well ordering ${\le}$.
	\end{itemize}
\end{theorem}

\section{Topological Spaces}

\begin{definition}
	Let $X$ be a nonempty set.
	A collection $\tau \subset \wp(X)$ is said to be a \emph{topology} if
	\begin{enumerate}
		\item $\emptyset,X \in \tau$,
		\item $\tau$ is closed under finite intersection, and
		\item $\tau$ is closed under arbitrary union.
	\end{enumerate}
	Then $(X,\tau)$ is called a \emph{topological space}.
	Elements of $\tau$ are called \emph{open sets}.
\end{definition}

\begin{example}\
		\begin{itemize}
		\item \emph{Trivial Topology:} $\tau \coloneqq \{ \emptyset, X\}$.
		\item \emph{Discrete Topology:} $\tau \coloneqq \wp(X)$.
		\item \emph{Standard Topology in $\bbR$:} $A$ is an open set (i.e., $A \in \tau$) iff for all $x \in A$, there exists $\epsilon > 0$, such that $(x-\epsilon,x+\epsilon) \subset A$.
		\item \emph{Sorgenfrey (Line) Topology in $\bbR$:} $A \in \tau$ iff for all $x \in A$, there exists $\epsilon > 0$, such that $[x,x+\epsilon) \subset A$.
		\item $\tau_r \coloneqq \{ (a,+\infty) : a \in \bbR \} \cup \{ \emptyset,\bbR \}$ and $\tau_l \coloneqq \{ (-\infty,a) : a \in \bbR \} \cup \{ \emptyset,\bbR\}$ are two topologies in $\bbR$.
	\end{itemize}
\end{example}

\begin{definition}
	Let $\tau_1$ and $\tau_2$ be topologies on $X$.
	$\tau_1$ is said to be \emph{coarser than} $\tau_2$ (i.e., $\tau_2$ is said to be \emph{finer than} $\tau_1$) if $\tau_1 \subset \tau_2$.
	The trivial topology is the coarsest and the discrete topology is the finest.
\end{definition}

\begin{definition}
	$x \in X$ is said to be an \emph{isolated point} if $\{x\}$ is open.
\end{definition}

\begin{remark}
	If $\{ \tau_\alpha \}_{\alpha \in \calI}$ is a collection of topologies on $X$, then
	\[
	\bigcap_{\alpha \in \calI} \tau_\alpha = \{ U \subset X : \forall \alpha \in \calI. (U \in \tau_\alpha) \}
	\]
	is still a topology on $X$.
\end{remark}

\begin{proposition}
	Let $X$ be a set and $\calF$ be a family of subsets of $X$.
	Then the smallest topology $\tau$ containing $\calF$ is given by arbitrary union of finite intersection of elements of $\calF \cup \{\emptyset,X\}$, i.e.,
	\[
	\tau = \left\{ \bigcup_{\alpha \in \calI} (U_1^\alpha \cap \cdots \cap U_{N(\alpha)}^\alpha) : N(\alpha) \in \bbN, U_i^\alpha \in \calF \cup \{ \emptyset, X \} \right\}.
	\]
	Moreover, we say that $\tau$ is \emph{generated} by $\calF$, and $\calF$ is a \emph{subbase} of $\tau$.
\end{proposition}

\begin{definition}
	Let $(X,\tau)$ be a topological space.
	$U \subset X$ is said to be a \emph{neighborhood} of $x \in X$ if $U$ is open and $x \in U$.
	Similarly, $U$ is said to be a neighborhood of $E \subset X$ if $U$ is open and $E \subset U$.
\end{definition}

\begin{definition}
	A family $\beta \subset \wp(X)$ is said to be a \emph{base} of the topology $\tau$ on $X$, if every open set can be written as a union of elements of $\beta$.
	
	Given $x \in X$, a family $\beta_x \subset \tau$ of neighborhoods of $x$ is said to be a \emph{local base} of $x$ if every neighborhood of $x$ contains an element of $\beta_x$.
\end{definition}

\begin{example}\
	\begin{itemize}
		\item Consider $(\bbR,\tau_{\mathrm{standard}})$. Then $\beta \coloneqq \{ (a,b) : a,b \in \bbQ, a \le b \}$ is a base.
		\item Consider $(\bbR,\tau_{\mathrm{sorg}})$.
		Then $\beta \coloneqq \{ [x,p) : x \in \bbR, x \le p, p \in \bbQ\}$ is a base.
		For all $x \in \bbR$, $\beta_x \coloneqq \{ [x,p) : x \le p, p \in \bbQ \}$ is a local base of $x$.
		\item Consider $(X,\tau_{\mathrm{discrete}})$. Then $\beta \coloneqq \{ \emptyset \} \cup \{ \{x \} : x \in X \}$ is a base.
	\end{itemize}
\end{example}

\begin{proposition}
	Suppose $X \neq \emptyset$ and $\beta \subset \wp(X)$.
	Then $\beta$ is a base for a topology on $X$ iff
	\begin{enumerate}
		\item $\emptyset \in \beta$,
		\item for all $x \in X$, there exists $B \in \beta$, such that $x \in B$, and
		\item for all $B_1,B_2 \in \beta$, $B_1 \cap B_2 \neq \emptyset$, then for all $x \in B_1 \cap B_2$, there exist $B_3 \in \beta$, such that $x \in B_3$ and $B_3 \subset B_1 \cap B_2$.
	\end{enumerate}
\end{proposition}

\section{Axioms of Countability}

\begin{definition}
	Let $(X,\tau)$ be a topological space.
	\begin{itemize}
		\item $(X,\tau)$ is said to be \emph{first countable} (i.e., $(X,\tau)$ satisfies the first axiom of countability), if every $x \in X$ admits a countable local base.
		\item $(X,\tau)$ is said to be \emph{second countable}, if it has a countable base.
	\end{itemize}
\end{definition}

\begin{remark}
	The second axiom of countability implies the first one, but the other direction does not hold in general.
\end{remark}

\section{Some Topologies}

\begin{definition}\
	\begin{itemize}
		\item \emph{Induced Topology:} Let $(X,\tau)$ be a topological space and $Y \subset X$.
		Then $\tau_Y \coloneqq \{ Y \cap U : U \in \tau \}$ is a topology on $Y$.
		\item \emph{Inverse Image Topology:} Let $f : X \to Y$ and $(Y,\tau_Y)$ be a topological space. Then $\tau_X \coloneqq \{ f^{-1}(V) : V \in \tau_Y \}$ is a topology on $X$.
		\item \emph{Direct Image Topology:} Let $f : X \to Y$ and $(X,\tau_X)$ be a topological space. Then $\tau_Y \coloneqq \{ V : f^{-1}(V) \in \tau_X \}$ is a topology on $Y$.
		\item \emph{Quotient Topology:} Let $(X,\tau)$ be a topological space and ${\sim}$ be an equivalence relation in $X$.
		For all $x \in X$, define $[x] \coloneqq \{ y \in X : y \sim x \}$ to be the equivalence class of $x$.
		Define $Y \coloneqq X /{\sim}$ and a projection map $p : X \to Y$ as $x \mapsto [x]$.
		The \emph{quotient topology} on $Y$ is the direct image topology of $\tau$ via $p$.
		\item \emph{Product Topology:} Let $(X,\tau_X)$ and $(Y,\tau_Y)$ be two topological spaces.
		The \emph{product topology} on $X \times Y$ is given by a subbase $\calF \coloneqq \{ U \times V : U \in \tau_X, V \in \tau_Y \}$.
	\end{itemize}
\end{definition}

\begin{remark}
	Let $(X,\tau_X)$ be a topological space and $f : X \to Y$.
	Let $\tau_Y$ be the direct image topology of $\tau_X$ via $f$ and $\tilde{\tau}_X$ be the inverse image topology of $\tau_Y$ via $f$.
	Then $\tilde{\tau}_X \subset \tau_X$.
\end{remark}

\begin{remark}
	Suppose the topology $(\bbR,\tau_{\mathrm{standard}})$.
	Define $f : \bbR \to \bbR$ as $x \mapsto 0$.
	Then the direct image topology $\tau_\bbR$ is $\{ V \subset \bbR : f^{-1}(V) \in \tau_\mathrm{standard} \}$, i.e., the discrete topology $\wp(\bbR)$.
	The inverse image topology is then $\tilde{\tau}_{\bbR} = \{ f^{-1}(V) : V \in \tau_\bbR \}$, i.e., the trivial topology $\{ \emptyset, \bbR \}$.
\end{remark}

\begin{definition}
	A property of topological spaces is said to be \emph{hereditary} if whenever a space has the property, so do all its subsets with induced topology.
\end{definition}

\begin{remark}
	The first/second axiom of countability are hereditary properties.
\end{remark}

\begin{remark}\
	\begin{itemize}
		\item Let $(X,\tau_X)$ be a topological space and $Y \subset X$.
		Let $\tau_Y$ be the induced topology on $Y$.
		Then $Y$ is open in $X$ iff $\tau_Y \subset \tau_X$.
		\item Let $(X,\tau_X)$ be a topological space and $i : Y \to X$ be an inclusion map (i.e., $y \mapsto y$).
		Then the induced topology on $Y$ is exactly the same as the inverse image topology via $i$.
		\item Let $(X,\tau_X)$ be a topological space and $f : X\to Y$ for some $Y$.
		Let $\tau_Y$ be the direct image topology via $f$ and $E \coloneqq Y \setminus f(X)$.
		Then the induced topology from $Y$ to $E$ is the discrete topology.
		\item Let $(X,\tau_X)$ and $(Y,\tau_Y)$ be two topological spaces.
		Define $\pi_X : X \times Y \to X$ as $(x,y) \mapsto x$ and $\pi_Y : X \times Y \to Y$ as $(x,y) \mapsto y$.
		Let $\sigma_1$ and $\sigma_2$ be the inverse image topology on $X \times Y$ via $\pi_X$ and $\pi_Y$, respectively.
		Let $\tau$ be the topology on $X \times Y$ generated by $\sigma_1 \cup \sigma_2$.
		then $\tau$ is the product topology.
	\end{itemize}	
\end{remark}

\section{Metric Spaces}

\begin{definition}
	A map $d : X \times X \to [0,+\infty)$ is said to be a \emph{metric} (\emph{distance}), if
	\begin{enumerate}
		\item 
		\begin{enumerate}
			\item $d(x,x) = 0$ for all $x \in X$,
			\item $x \neq y$ implies $d(x,y) > 0$ for all $x,y \in X$,
		\end{enumerate}
		\item $d(x,y) = d(y,x)$ for all $x,y \in X$, and
		\item $d(x,z) \le d(x,y) + d(y,z)$ for all $x,y,z \in X$.
	\end{enumerate}
	Then $(X,d)$ is said to be a \emph{metric space}.
\end{definition}

\begin{definition}
	Let $(X,d)$ be a metric space.
	The \emph{ball} centered at $x in X$ with radius $r > 0$, written $B(x,r)$, is defined as $\{ y \in X : d(x,y) < r\}$.
\end{definition}

\begin{lemma}
	$\beta \coloneqq \{ B(x,r) : x \in X, r > 0 \} \cup \{ \emptyset \}$ is a base for a topology.
	In other words, every metric space $(X,d)$ admits a natural topology $\tau_d$.
\end{lemma}

\begin{example}\
	\begin{itemize}
		\item Let $X = \bbR$ and $d(x,y) \coloneqq |x-y|$ be a metric in $X$.
		Then $B(x,\epsilon) = (x-\epsilon,x+\epsilon)$, i.e., the metric space $(X,d)$ admits the standard topology.
		\item Suppose $X$ is a set.
		Define $\tilde{L}_\infty(X) \coloneqq \{ f :  X \to \bbR : f~\text{is bounded} \}$.
		Then $d_\infty(f,g) \coloneqq \sup_{x \in X} |f(x) - g(x)|$ is a metric.
		\item Consider $[a,b] \subset \bbR$ and $C([a,b]) = \{ f : [a,b] \to \bbR : f~\text{is continuos}\}$.
		Then $d(f,g) \coloneqq \max_{x \in [a,b]} |f(x)-g(x)|$ is a metric.
		\item Consider the space $\bbR^N$.
		Define
		\begin{eqnarray*}
			d_1(x,y) & \coloneqq & \sqrt{\sum_{i=1}^N (x_i-y_i)^2} \\
			d_2(x,y) & \coloneqq & \sum_{i=1}^N |x_i - y_i| \\
			d_3(x,y) & \coloneqq & \max_{i=1,\cdots,N} |x_i - y_i|
		\end{eqnarray*}
		then the topologies associated to these metrics coincide.
	\end{itemize}
\end{example}

\begin{remark}
	Suppose $(X,d)$ is a metric space, $x \in X$ and $r > 0$.
	In general $\overline{B(x,r)} \neq \{y \in X : d(x,y) \le r \}$.
\end{remark}

\begin{definition}
	Let $(X,d)$ be a metric space and $A \subset X$.
	The \emph{diameter} of $A$, written $\mathrm{diam}(A)$, is defined as $\sup \{ d(x,y) : x,y \in A$.
	$A$ is said to be \emph{bounded} if $\mathrm{diam}(A) < +\infty$.
\end{definition}

\begin{proposition}
	Let $(X,d)$ be a metric space.
	Define $d_1(x,y) \coloneqq \frac{d(x,y)}{1 + d(x,y)}$.
	Then $(X,d_1)$ is a bounded metric space and $\tau_d = \tau_{d_1}$.
\end{proposition}

\begin{lemma}
	If $d_1,d_2$ are metrics on $X$ and $d_1 \le d_2$, then $\tau_{d_1} \subset \tau_{d_2}$.
\end{lemma}

\begin{remark}
	Consider $C((0,1)) = \{ f : (0,1) \to \bbR : f~\text{is continuous} \}$.
	Define $K_n \coloneqq [\frac{1}{n}, 1-\frac{1}{n}]$ for $n \in \bbN^+$.
	Then $\bigcup_{n=1}^\infty K_n = (0,1)$.
	Define
	\begin{eqnarray*}
		d(f,g) & \coloneqq & \max_{n \in \bbN^+} \left\{ \frac{1}{2^n} \max_{x \in K_n} \frac{|f(x)-g(x)|}{1+|f(x)-g(x)|} \right\} \\
		\tilde{d}(f,g) & \coloneqq & \sup_{x \in (0,1)} \frac{|f(x)-g(x)|}{1+|f(x)-g(x)|}
	\end{eqnarray*}
	to be two metrics on $C(0,1)$.
	Then $\tau_d \subset \tau_{\tilde{d}}$ and $\tau_d \neq \tau_{\tilde{d}}$.
\end{remark}

\begin{definition}
	Let $\rho$ be a pseudo-metric (i.e., a metric without the property $x \neq y \implies d(x,y) > 0$) on $X$.
	Define an equivalence relation ${\sim}$ as $x \sim y$ iff $\rho(x,y) = 0$.
	Then $(Y,d)$ with $Y \coloneqq X / {\sim}$ and $d([x],[y]) \coloneqq \rho(x,y)$ is a metric space, named the \emph{quotient metric space}.
\end{definition}

\bibliographystyle{abbrv}
%\bibliography{db}
\end{document}
