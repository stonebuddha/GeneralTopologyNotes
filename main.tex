\documentclass{techreport}

\newcommand{\Omit}[1]{}
\newcommand{\diw}[1]{{\color{ACMRed} DW: #1}}

\sloppy
\begin{document}
\title{21-651: General Topology}
\author[Di Wang]{Di Wang (diw3)}
\maketitle

\section{Sets and Ordering}

\begin{definition}\label{De:Equipotent}
	Two sets $A,B$ are said to be \emph{equipotent}, written as $A \sim B$, if there exists a bijection between them.
\end{definition}

\begin{remark}\label{Rem:EquipotenceIsEquivalence}
	Equipotence is an equivalence relation.
\end{remark}

\begin{definition}\label{De:Countability}\
	\begin{itemize}
		\item A set is said to be \emph{finite} if it is equipotent to a set $\{1,\cdots,n\} \subseteq \bbN$ for some $n \in \bbN$.
		\item A set is said to be \emph{countable} if it is equipotent to some subset of $\bbN$.
		\item A set is said to be \emph{uncountable} if it is not countable.
	\end{itemize}
\end{definition}

\begin{remark}\label{Rem:RIsUncountable}
	$\bbR$ is uncountable.
\end{remark}

\begin{theorem}[Cantor-Bernstein-Schr{\"o}der]\label{The:CantorBernsteinSchroder}
	If there exist injections $f:A \to B$ and $g : B \to A$, then $A \sim B$.
\end{theorem}

\begin{definition}\label{De:PowerSet}
	Let $X$ be a set.
	The \emph{power set} of $X$, written as $\wp(X)$, is the collection of all subsets of $X$.
\end{definition}

\begin{lemma}[Cantor]\label{Lem:CantorPowerSet}
	Let $X$ be a set. There is no surjection from $X$ to $\wp(X)$.
\end{lemma}

\begin{remark}\label{Rem:PowerSetAndIndicators}
	Let $X$ be a set. Then
	\begin{equation*}
		\wp(X) \sim \{0,1\}^X \coloneqq \set{ f : X \to \{0,1\} }.
	\end{equation*}
\end{remark}

\begin{definition}\label{De:PartialOrder}
	A \emph{partial order} on a set $X$ is a binary relation on $X$, written as ${\le}$, if
	\begin{itemize}
		\item (reflexivity): $x \le x$ for all $x \in X$,
		\item (antisymmetry): $x \le y$ and $y \le x$ imply $x = y$ for all $x,y \in X$, and
		\item (transitivity): $x \le y$ and $y \le z$ imply $x \le z$ for all $x,y,z \in X$.
	\end{itemize}
	Then $(X,{\le})$ is said to be a \emph{partial ordering}, or \emph{poset}.
\end{definition}

\begin{example}\label{Exa:RSubsetIsPoset}
	Let $X \coloneqq \wp(\bbR)$.
	Then $A \le B \coloneqq A \subseteq B$ is a partial order on $X$.
\end{example}

\begin{definition}\label{De:LinearOrder}
	A partial ordering $(X,{\le})$ is a \emph{linear ordering} if for all $x,y \in X$, we have either $x \le y$ or $y \le x$.
\end{definition}

\begin{definition}\label{De:BoundingElementsInPoset}
	Let $(X,{\le})$ be a partial ordering.
	Let $E \subseteq X$.
	\begin{itemize}
		\item $x \in X$ is said to be an \emph{upper bound} of $E$ if for all $y \in E$, we have $y \le x$.
		\item $x \in E$ is said to be a \emph{maximal element} of $E$ if there does not exist $y \in E$ such that $x < y$.
		\item If $x \in E$ and for all $y \in E$ we have $y \le x$, then $x$ is the \emph{greatest element} of $E$.
	\end{itemize}
	\emph{Lower bounds}, \emph{minimal elements}, and \emph{least elements} are defined in a similar way.
\end{definition}

\begin{definition}\label{De:WellOrdering}
	A poset $(X,{\le})$ is said to be \emph{well-ordered} if it is linearly ordered and every nonempty subset of $X$ has the least element.
\end{definition}

\begin{theorem}\label{The:EquivalentZFC}
	The following three are equivalent:
	\begin{itemize}
		\item \emph{Axiom of Choice:} Given a collection $\calA$ of nonempty disjoint sets, there exists a set $C$ such that (i) $C \subseteq \bigcup \calA$, and (ii) for all $A \in \calA$, $C \cap A$ has exactly one element.
		\item \emph{Zorn's Lemma:} Suppose $(X,{\le})$ is a poset. If every chain in $X$ has an upper bound, then $X$ has a maximal element.
		\item \emph{Well Ordering Lemma (Zermelo's Theorem):} Every nonempty set $X$ admits a well ordering ${\le}$.
	\end{itemize}
\end{theorem}

\section{Topological Spaces}

\begin{definition}\label{De:Topology}
	Let $X$ be a nonempty set.
	A collection $\tau \subseteq \wp(X)$ is said to be a \emph{topology} if
	\begin{enumerate}
		\item $\emptyset,X \in \tau$,
		\item $\tau$ is closed under finite intersection, and
		\item $\tau$ is closed under arbitrary union.
	\end{enumerate}
	Then $(X,\tau)$ is said to be a \emph{topological space}.
	Elements of $\tau$ are said to be \emph{open sets}.
\end{definition}

\begin{example}\label{Exa:BasicTopologies}\
		\begin{itemize}
		\item \emph{Trivial Topology:} $\tau \coloneqq \{ \emptyset, X\}$.
		\item \emph{Discrete Topology:} $\tau \coloneqq \wp(X)$.
		\item \emph{Standard Topology in $\bbR$:} $A$ is an open set (i.e., $A \in \tau$) iff for all $x \in A$, there exists $\epsilon > 0$, such that $(x-\epsilon,x+\epsilon) \subseteq A$.
		\item \emph{Sorgenfrey (Line) Topology in $\bbR$:} $A \in \tau$ iff for all $x \in A$, there exists $\epsilon > 0$, such that $[x,x+\epsilon) \subseteq A$.
		\item $\tau_r \coloneqq \{ (a,+\infty) : a \in \bbR \} \cup \{ \emptyset,\bbR \}$ and $\tau_l \coloneqq \{ (-\infty,a) : a \in \bbR \} \cup \{ \emptyset,\bbR\}$ are two topologies in $\bbR$.
	\end{itemize}
\end{example}

\begin{definition}\label{De:TopComparison}
	Let $\tau_1$ and $\tau_2$ be topologies on $X$.
	$\tau_1$ is said to be \emph{coarser than} $\tau_2$ (i.e., $\tau_2$ is said to be \emph{finer than} $\tau_1$) if $\tau_1 \subseteq \tau_2$.
	The trivial topology is the coarsest and the discrete topology is the finest.
\end{definition}

\begin{definition}\label{De:IsolatedPoint}
	$x \in X$ is said to be an \emph{isolated point} if $\{x\}$ is open.
\end{definition}

\begin{remark}\label{Rem:IntersectTopStillTop}
	If $\{ \tau_\alpha \}_{\alpha \in \calI}$ is a collection of topologies on $X$, then
	\begin{equation*}
		\bigcap_{\alpha \in \calI} \tau_\alpha = \set{ U \subseteq X : \Forall{\alpha \in \calI} U \in \tau_\alpha }
	\end{equation*}
	is still a topology on $X$.
\end{remark}

\section{Bases and Subbases}

\begin{proposition}\label{Prop:Subbase}
	Let $X$ be a set and $\calF$ be a family of subsets of $X$.
	Then the smallest topology $\tau$ containing $\calF$ is given by arbitrary union of finite intersection of elements of $\calF \cup \{\emptyset,X\}$, i.e.,
	\begin{equation*}
		\set{ \bigcup_{\alpha \in \calI} \p{U_1^\alpha \cap \cdots \cap U_{N(\alpha)}^\alpha} : N(\alpha) \in \bbN, U_i^\alpha \in \calF \cup \{ \emptyset, X \} }.
	\end{equation*}
	Moreover, we say that $\tau$ is \emph{generated} by $\calF$, and $\calF$ is a \emph{subbase} of $\tau$.
\end{proposition}
\begin{proof}
	Define $\tau$ to be the smallest topology containing $\calF$ and $\tau'$ to be the topology constructed by arbitrary union of finite intersection of elements of $\calF$ and $\emptyset$.
	Then because $\tau$ is the smallest topology containing $\calF$, we know that $\tau \subseteq \tau'$.
	Then it suffices to show that $\tau' \subseteq \tau$.
	Pick an arbitrary
	\begin{equation}\label{Eq:UnionIntersect}
	\bigcup_{\alpha \in \calJ} \p{ U_1^\alpha \cap \cdots \cap U_{N(\alpha)}^\alpha }
	\end{equation}
	from $\tau'$.
	Because $U_i^\alpha$ are elements of $\calF \cup \{ \emptyset,X\}$, and indeed $\tau \supseteq \calF \cup \{ \emptyset,X\}$, we know that $U_i^\alpha \in \tau$.
	By definition of topologies, we know \eqref{UnionIntersect} is an open set, i.e., it is contained in $\tau$.
	Therefore we conclude that $\tau' \subseteq \tau$.
\end{proof}

\begin{definition}\label{De:Neighborhoods}
	Let $(X,\tau)$ be a topological space.
	$U \subseteq X$ is said to be a \emph{neighborhood} of $x \in X$ if $U$ is open and $x \in U$.
	Similarly, $U$ is said to be a neighborhood of $E \subseteq X$ if $U$ is open and $E \subseteq U$.
\end{definition}

\begin{definition}\label{De:BaseLocalBase}
	A family $\beta \subseteq \tau$ is said to be a \emph{base} of the topology $\tau$ on $X$, if every open set can be written as a union of elements of $\beta$.
	
	Given $x \in X$, a family $\beta_x \subseteq \{ U \in \tau : x \in U\}$ of neighborhoods of $x$ is said to be a \emph{local base} of $x$ if every neighborhood of $x$ contains an element of $\beta_x$.
\end{definition}

\begin{example}\label{Exa:BaseLocalBase}\
	\begin{itemize}
		\item Consider $(\bbR,\tau_{\mathrm{standard}})$. Then $\beta \coloneqq \{ (a,b) : a,b \in \bbQ, a \le b \}$ is a base.
		\item Consider $(\bbR,\tau_{\mathrm{sorg}})$.
		Then $\beta \coloneqq \{ [x,p) : x \in \bbR, x \le p, p \in \bbQ\}$ is a base.
		For all $x \in \bbR$, $\beta_x \coloneqq \{ [x,p) : x \le p, p \in \bbQ \}$ is a local base of $x$.
		\item Consider $(X,\tau_{\mathrm{discrete}})$. Then $\beta \coloneqq \{ \emptyset \} \cup \{ \{x \} : x \in X \}$ is a base.
	\end{itemize}
\end{example}

\begin{proposition}\label{Prop:BaseStructure}
	Suppose $X \neq \emptyset$ and $\beta \subseteq \wp(X)$.
	Then $\beta$ is a base for a topology on $X$ iff
	\begin{enumerate}
		\item $\emptyset \in \beta$,
		\item for all $x \in X$, there exists $B \in \beta$, such that $x \in B$, and
		\item for all $B_1,B_2 \in \beta$, $B_1 \cap B_2 \neq \emptyset$, then for all $x \in B_1 \cap B_2$, there exist $B_3 \in \beta$, such that $x \in B_3$ and $B_3 \subseteq B_1 \cap B_2$.
	\end{enumerate}
\end{proposition}
\begin{proof}
	\begin{itemize}
		\item For the ${\implies}$ direction, let $\beta$ is a base for $\tau$.
		\begin{itemize}
			\item WTS (1): Trivial by $\emptyset \in \tau$.
			\item WTS (2): Trivial by $X \in \tau$.
			\item WTS (3): Fix $B_1$ and $B_2$ and let $x \in B_1 \cap B_2$.
			Because $B_1,B_2 \in \beta \subseteq \tau$, i.e., $B_1$ and $B_2$ are open sets, we know that $B_1 \cap B_2$ is also open.
			By definition of bases, we know there exists a collection $\{U_\alpha \in \beta\}_{\alpha \in \calI}$ such that
			\[
			B_1 \cap B_2 = \bigcup_{\alpha \in \calI} U_\alpha.
			\]
			Because $x \in B_1 \cap B_2$, we know that there exists $\alpha_0 \in \calI$ such that $x \in U_{\alpha_0}$.
			Set $B_3 \coloneqq U_{\alpha_0} \in \beta$.
			Then we conclude that $x \in B_3$ and $B_3 \subseteq B_1 \cap B_2$.
		\end{itemize}
		
		\item For the ${\impliedby}$ direction, let $\tau$ be the collection constructed by arbitrary union of elements of $\beta$.
		Then indeed $\beta \subseteq \tau$.
		We claim that $\tau$ is a topology.
		\begin{itemize}
			\item WTS \defref{Topology}(1): On the one hand, (1) implies that $\emptyset \in \tau$.
			On the other hand, for each $x \in X$, by (2) we know there exists $B_x \in \beta$ such that $x \in B_x$.
			Therefore
			\[
			X = \bigcup_{x \in X} B_x \in \tau.
			\]
			
			\item WTS \defref{Topology}(2):
			Suppose $U_1,\cdots,U_n \in \tau$.
			It suffices to show that $U_1 \cap \cdots \cap U_n \in \tau$.
			For each $i=1,\cdots,n$, $U_i \in \tau$ implies that
			\[
			U_i = \bigcup \calB_i \quad \text{for some} \quad \calB_i \subseteq \beta.
			\]
			Let $x \in U_1 \cap \cdots \cap U_n$.
			Then for each $i=1,\cdots,n$, there exists $B_x^{(i)} \in \calB_i$ such that $x \in B_x^{(i)}$.
			Therefore $x \in B_x^{(1)} \cap \cdots \cap B_x^{(n)}$.
			By induction with the property (3) we know that there exists $B_x \in \beta$ such that $B_x \subseteq B_x^{(1)} \cap \cdots \cap B_x^{(n)}$.
			Therefore, we conclude that
			\[
			U_1 \cap \cdots \cap U_n = \bigcup_{x \in U_1 \cap \cdots \cap U_n} B_x \in \tau.
			\]
			
			\item WTS \defref{Topology}(3):
			Suppose $\{U_\alpha \in \tau\}_{\alpha \in \calI}$.
			It suffices to show that $\bigcup_{\alpha \in \calI} U_\alpha \in \tau$.
			For each $\alpha \in \calI$, $U_\alpha \in \tau$ implies that
			\[
			U_\alpha = \bigcup \calB_\alpha \quad \text{for some} \quad \calB_\alpha \subseteq \beta.
			\]
			Then $\bigcup_{\alpha \in \calI} U_\alpha = \bigcup_{\alpha \in \calI} \bigcup \calB_\alpha$ is indeed contained in $\tau$.
		\end{itemize}
	\end{itemize}
\end{proof}

\section{Countability Axioms}

\begin{definition}\label{De:AxiomsOfCountability}
	Let $(X,\tau)$ be a topological space.
	\begin{itemize}
		\item $(X,\tau)$ is said to be \emph{first countable} (i.e., $(X,\tau)$ satisfies the first axiom of countability), if every $x \in X$ admits a countable local base.
		\item $(X,\tau)$ is said to be \emph{second countable} (i.e., $(X,\tau)$ satisfies the second axiom of countability), if it has a countable base.
	\end{itemize}
\end{definition}

\begin{proposition}\label{Prop:SndAxCountImplyFstAx}
	If a topological space $(X,\tau)$ is second countable, then it is first countable.
\end{proposition}
\begin{proof}
	Let $\beta = \{ B_n\}_{n \in \bbN} \subseteq \tau$ be a base.
	We claim that for every $x \in X$, the collection $\beta_x \coloneqq \{ B_n \in \beta : x \in B_n\}$ is a local base of $x$.
	Suppose $U$ is a neighborhood of $x$. Then there exists $\calI \subseteq \bbN$ such that
	\[
	U = \bigcup_{n \in \calI} B_n.
	\]
	On the other hand, $x \in U$ implies that there exists $n_0 \in \calI$ such that $x \in B_{n_0}$.
	By definition of $\beta_x$ we know that indeed $B_{n_0} \in \beta_x$.
	Thus we conclude the proof by $U \supseteq B_{n_0}$.
\end{proof}

\begin{proposition}\label{Prop:SorgFstCountNotSnd}
	The Sorgenfrey line topology in $\bbR$ is first countable but not second countable.
\end{proposition}
\begin{proof}
	\begin{itemize}
		\item For each $x \in \bbR$, $\beta_x \coloneqq \{ [x,x+\nicefrac{1}{n}) : n \in \bbN^+ \}$ is a countable local base of $x$. Therefore $(\bbR,\tau_{\mathrm{sorg}})$ is first countable.
		\item We claim that if $\beta \subseteq \tau_{\mathrm{sorg}}$ is a base, the map $\Phi : \beta \to \bbR \cup \{ - \infty \}$ defined as $B \mapsto \inf B$ is onto.
		Suppose $x \in \bbR$.
		Because $[x,x+1) \in \tau_{\mathrm{sorg}}$, we know that there exists a collection $\{ U_\alpha\}_{\alpha \in \calI} \subseteq \beta$ such that
		\[
		[x,x+1) = \bigcup_{\alpha \in \calI} U_\alpha.
		\]
		Therefore there exists $\alpha_0 \in \calI$ such that $x \in U_{\alpha_0}$.
		On the other hand, $U_{\alpha_0} \subseteq [x,x+1)$ implies that $\inf U_{\alpha_0} = x$.
		Therefore $\Phi(U_\alpha) = x$.
		Because $x$ is arbitrarily chosen from $\bbR$, and by the fact that $\Phi(\emptyset) = -\infty$, we conclude that $\Phi$ is onto, and $\beta$ cannot be countable.
	\end{itemize}
\end{proof}

\section{Topological Constructions}

\begin{definition}\label{De:InducedTop}
	Let $(X,\tau)$ be a topological space and $Y \subseteq X$.
	Then $\tau_Y \coloneqq \{ Y \cap U : U \in \tau \}$ is said to be the \emph{induced topology} on $Y$ from $(X,\tau)$.
\end{definition}

\begin{definition}\label{De:InverseImageTop}
	Let $(Y,\tau_Y)$ be a topological space and $f : X \to Y$.
	Then $\tau_X \coloneqq \{ f^{-1}(V) : V \in \tau_Y \}$ is said to be the \emph{inverse image topology} on $X$ via $f$.
\end{definition}

\begin{definition}\label{De:DirectImageTop}
	Let $(X,\tau_X)$ be a topological space and $f : X \to Y$.
	Then $\tau_Y \coloneqq \{ V \subseteq Y : f^{-1}(V) \in \tau_X \}$ is said to be the \emph{direct image topology} on $Y$ via $f$.
\end{definition}

\begin{definition}\label{De:QuotientTop}
	Let $(X,\tau)$ be a topological space and ${\sim}$ be an equivalence relation in $X$.
	For all $x \in X$, define $[x] \coloneqq \{ y \in X : y \sim x \}$ to be the equivalence class of $x$. 
	Define $Y \coloneqq X /{\sim}$ and a projection map $p : X \to Y$ as $x \mapsto [x]$.
	Then the direct image topology of $\tau$ via $p$ is said to be the \emph{quotient topology} on $Y$.
\end{definition}

\begin{definition}\label{De:ProductTop}
	Let $(X,\tau_X)$ and $(Y,\tau_Y)$ be two topological spaces.
	Then the topology generated by $\calF \coloneqq \{ U \times V : U \in \tau_X, V \in \tau_Y \}$ is said to be the product topology on $X \times Y$.
\end{definition}

\begin{proposition}\label{Prop:DirectThenInverseIsWeaker}
	Let $(X,\tau_X)$ be a topological space and $f : X \to Y$.
	Let $\tau_Y$ be the direct image topology of $\tau_X$ via $f$ and $\tilde{\tau}_X$ be the inverse image topology of $\tau_Y$ via $f$.
	Then $\tilde{\tau}_X \subseteq \tau_X$.
\end{proposition}
\begin{proof}
	By definition we have
	\[
	\tilde{\tau}_X \coloneqq \set{ f^{-1}(V) : V \in \tau_Y }.
	\]
	For every $U \in \tilde{\tau}_X$, there exists $V \in \tau_Y$ such that $U = f^{-1}(V)$.
	On the other hand, by definition we have
	\[
	\tau_Y \coloneqq \{ V \subseteq Y : f^{-1}(V) \in \tau_X \}.
	\]
	Thus indeed we have $U = f^{-1}(V) \in \tau_X$.
\end{proof}

\begin{example}\label{Exa:DirectTheInverseIsWeaker}
	Consider the topology $(\bbR,\tau_{\mathrm{standard}})$.
	Define $f : \bbR \to \bbR$ as $x \mapsto 0$.
	Then the direct image topology $\tau_\bbR$ is $\{ V \subseteq \bbR : f^{-1}(V) \in \tau_\mathrm{standard} \}$, i.e., the discrete topology $\wp(\bbR)$.
	The inverse image topology is then $\tilde{\tau}_{\bbR} = \{ f^{-1}(V) : V \in \tau_\bbR \}$, i.e., the trivial topology $\{ \emptyset, \bbR \}$.
\end{example}

\begin{definition}\label{De:HereditaryProperty}
	A property of topological spaces is said to be \emph{hereditary} if whenever a space has the property, so do all its subsets with induced topology.
\end{definition}

\begin{remark}\label{Rem:AxCountHereditary}
	The first/second axiom of countability are hereditary properties.
\end{remark}

\begin{remark}\label{Rem:SomeResultsAboutTopologies}\
	\begin{itemize}
		\item Let $(X,\tau_X)$ be a topological space and $Y \subseteq X$.
		Let $\tau_Y$ be the induced topology on $Y$.
		Then $Y$ is open in $X$ iff $\tau_Y \subseteq \tau_X$.
		\item Let $(X,\tau_X)$ be a topological space and $Y \subseteq X$.
		Let $i : Y \to X$ be the inclusion map (i.e., $y \mapsto y$).
		Then the induced topology on $Y$ is exactly the same as the inverse image topology via $i$.
		\item Let $(X,\tau_X)$ be a topological space and $f : X\to Y$ for some $Y$.
		Let $\tau_Y$ be the direct image topology via $f$ and $E \coloneqq Y \setminus f(X)$.
		Then the induced topology from $Y$ to $E$ is the discrete topology.
		\item Let $(X,\tau_X)$ and $(Y,\tau_Y)$ be two topological spaces.
		Define $\pi_X : X \times Y \to X$ as $(x,y) \mapsto x$ and $\pi_Y : X \times Y \to Y$ as $(x,y) \mapsto y$.
		Let $\sigma_1$ and $\sigma_2$ be the inverse image topology on $X \times Y$ via $\pi_X$ and $\pi_Y$, respectively.
		Let $\tau$ be the topology on $X \times Y$ generated by $\sigma_1 \cup \sigma_2$.
		Then $\tau$ is exactly the product topology on $X \times Y$.
	\end{itemize}	
\end{remark}

\section{Metric Spaces}

\begin{definition}\label{De:Metrics}
	A map $d : X \times X \to [0,+\infty)$ is said to be a \emph{metric} (\emph{distance}), if
	\begin{enumerate}
		\item 
		\begin{enumerate}
			\item $d(x,x) = 0$ for all $x \in X$,
			\item $x \neq y$ implies $d(x,y) > 0$ for all $x,y \in X$,
		\end{enumerate}
		\item $d(x,y) = d(y,x)$ for all $x,y \in X$, and
		\item $d(x,z) \le d(x,y) + d(y,z)$ for all $x,y,z \in X$.
	\end{enumerate}
	Then $(X,d)$ is said to be a \emph{metric space}.
\end{definition}

\begin{definition}\label{De:OpenBalls}
	Let $(X,d)$ be a metric space.
	The \emph{ball} centered at $x \in X$ with radius $r > 0$, written as $B(x,r)$, is defined as $\{ y \in X : d(x,y) < r\}$.
\end{definition}

\begin{lemma}\label{Lem:MetricInduceTop}
	$\beta \coloneqq \{ B(x,r) : x \in X, r > 0 \} \cup \{ \emptyset \}$ is a base for a topology.
	In other words, every metric space $(X,d)$ admits a natural topology $\tau_d$ with $\beta$ as its base.
\end{lemma}
\begin{proof}
	By \propref{BaseStructure}(3) it is sufficient to show that for all $B_1,B_2 \in \beta$ satisfying $B_1 \cap B_2 \neq \emptyset$, then for all $x \in B_1 \cap B_2$, there exists $B_3 \in \beta$ such that $x \in B_3$ and $B_3 \subseteq B_1 \cap B_2$.
	Let $B_1 = B(x_1,r_1)$ and $B_2 = B(x_2,r_2)$.
	For all $x \in B_1 \cap B_2$, define $r \coloneqq \min \{ r_1 - d(x,x_1), r_2 - d(x,x_2) \} > 0$.
	Define $B_3 \coloneqq B(x,r) \in \beta$.
	Then $x \in B_3$ and we claim that $B_3 \subseteq B_1 \cap B_2$.
	For every $x_o \in B_3$, we have
	\begin{align*}
		& d(x_o,x) < r \\
		\implies & d(x_o,x) < \min \{ r_1 - d(x,x_1), r_2 - d(x,x_2) \} \\
		\implies & d(x_o,x) + d(x,x_1) < r_1 \wedge d(x_o,x)+d(x,x_2) < r_2 \\
		\implies & d(x_o,x_1) < r_1 \wedge d(x_o,x_2) < r_2 \\
		\implies & x_o \in B_1 \wedge x_o \in B_2
	\end{align*}
	and indeed we conclude that $x_o \in B_1 \cap B_2$.
\end{proof}

\begin{example}\label{Exa:MetricSpaces}\
	\begin{itemize}
		\item Let $X = \bbR$ and $d(x,y) \coloneqq \abs{x-y}$ be a metric in $X$.
		Then $B(x,\epsilon)$ is the open interval $(x-\epsilon,x+\epsilon)$, i.e., the metric space $(X,d)$ admits the standard topology in $\bbR$.
		\item Suppose $X$ is a set.
		Define $\tilde{L}_\infty(X) \coloneqq \{ f :  X \to \bbR : f~\text{is bounded} \}$.
		Then $d_\infty(f,g) \coloneqq \sup_{x \in X} \abs{f(x) - g(x)}$ is a metric.
		\item Consider $[a,b] \subseteq \bbR$ and $C([a,b]) = \{ f : [a,b] \to \bbR : f~\text{is continuos}\}$.
		Then $d(f,g) \coloneqq \max_{x \in [a,b]} \abs{f(x)-g(x)}$ is a metric.
		\item Consider the space $\bbR^N$.
		Define
		\begin{align*}
			d_1(x,y) & \coloneqq \sqrt{\sum_{i=1}^N (x_i-y_i)^2} \\
			d_2(x,y) & \coloneqq \sum_{i=1}^N \abs{x_i - y_i} \\
			d_3(x,y) & \coloneqq \max_{i=1,\cdots,N} \abs{x_i - y_i}
		\end{align*}
		then the topologies associated to these metrics coincide.
		Indeed, they are all standard topology in $\bbR^N$, or the \emph{Euclidean topology}.
	\end{itemize}
\end{example}

\begin{remark}\label{Rem:ClosureOfBallsNotClosedBalls}
	Suppose $(X,d)$ is a metric space, $x \in X$ and $r > 0$.
	In general $\overline{B(x,r)} \neq \{y \in X : d(x,y) \le r \}$.
\end{remark}

\begin{definition}\label{De:Diameter}
	Let $(X,d)$ be a metric space and $A \subseteq X$.
	The \emph{diameter} of $A$, written as $\mathrm{diam}(A)$, is defined as $\sup \{ d(x,y) : x,y \in A$.
	$A$ is said to be \emph{bounded} if $\mathrm{diam}(A) < +\infty$.
\end{definition}

\begin{proposition}\label{Prop:MetricOpenFindABall}
	Let $(X,d)$ be a metric space. Then $U \in \tau_d$ iff for all $x \in U$, there exists $r >0$ such that $B(x,r) \subseteq U$.
\end{proposition}
\begin{proof}
	\begin{itemize}
		\item For the ${\implies}$ direction, let $U \in \tau_d$.
		Then there exists a collection $\{B(x_\alpha,r_\alpha)\}_{\alpha \in \calI}$ of balls such that
		\[
		U = \bigcup_{\alpha \in \calI} B(x_\alpha,r_\alpha).
		\]
		For every $x \in U$, then, there exists $\alpha \in \calI$ such that $x \in B(x_\alpha,r_\alpha)$.
		Let $s_\alpha \coloneqq d(x,x_\alpha) < r_\alpha$.
		We claim that $B(x,r_\alpha-s_\alpha) \subseteq U$.
		For every $x_o \in B(x,r_\alpha-s_\alpha)$, we have
		\[
		d(x_o,x_\alpha) \le d(x_o,x) + d(x,x_\alpha) < (r_\alpha - s_\alpha) + s_\alpha = r_\alpha
		\]
		hence $x_o \in B(x_\alpha,r_\alpha)$ and indeed $x_o \in U$.
		Therefore we conclude that $B(x,r_\alpha-s_\alpha) \subseteq U$.
		
		\item Suppose for every $x \in U$, there exists $r_x > 0$ such that $B(x,r_x) \subseteq U$.
		Then
		\[
		U = \bigcup_{x \in U} B(x,r_x)
		\]
		is trivially open in $X$.
	\end{itemize}
\end{proof}

\begin{proposition}\label{Prop:MetricCanAlwaysBounded}
	Let $(X,d)$ be a metric space.
	Define $d_1(x,y) \coloneqq \nicefrac{d(x,y)}{(1 + d(x,y))}$.
	Then $(X,d_1)$ is a bounded metric space and $\tau_d = \tau_{d_1}$.
\end{proposition}
\begin{proof}
	For all $x,y \in X$, we have $d_1(x,y) = \nicefrac{d(x,y)}{(1+d(x,y))} < 1$, thus $\mathrm{diam}(X) \le 1$ and $X$ is trivially bounded.
	Obverse that for any $R \in (0,1)$, and for all $x, y \in X$,
	\begin{align*}
		& \frac{d(x,y)}{1+d(x,y)} < R \\
		\iff & d(x,y) < R + R \cdot d(x,y) \\
		\iff & d(x,y) < \frac{R}{1-R}.
	\end{align*}
	Thus we know that for all $x \in X$ and $R \in (0,1)$,
	\[
	B_{d_1}(x,R) = B_d\p{x, \frac{R}{1-R} }.
	\]
	Then by \propref{MetricOpenFindABall} we conclude that $d$ and $d_1$ yield the same topology.
\end{proof}

\begin{lemma}\label{Lem:SmallerMetricIsWeaker}
	If $d_1,d_2$ are metrics on $X$ and $d_1 \le d_2$, then $\tau_{d_1} \subseteq \tau_{d_2}$.
\end{lemma}
\begin{proof}
	Let $U \in \tau_{d_1}$.
	We want to show that $U \in \tau_{d_2}$ and by \propref{MetricOpenFindABall} it suffices to show that for every $x \in U$, there exists $r >0$ such that $B_{d_2}(x,r) \subseteq U$.
	Suppose $x_0 \in U$.
	Also by \propref{MetricOpenFindABall} we know that there exists $r_0 > 0$ such that $B_{d_1}(x_0,r_0) \subseteq U$.
	For any $y \in B_{d_2}(x_0,r_0)$, we know that $d_1(x_0,y) \le d_2(x_0,y) < r_0$ thus $y \in B_{d_1}(x_0,r_0)$.
	Therefore we have
	\[
	B_{d_2}(x_0,r_0) \subseteq B_{d_1}(x_0,r_0) \subseteq U.
	\]
\end{proof}

\begin{proposition}\label{Prop:SmallerMetricIndeedWeaker}
	Consider $C((0,1)) = \{ f : (0,1) \to \bbR : f~\text{is continuous} \}$.
	Define $K_n \coloneqq \sq{\frac{1}{n}, 1-\frac{1}{n}}$ for $n \in \bbN^+$.
	Then $\bigcup_{n=1}^\infty K_n = (0,1)$.
	Define
	\begin{align*}
		d(f,g) & \coloneqq \max_{n \in \bbN^+} \set{ \frac{1}{2^n} \max_{x \in K_n} \frac{\abs{f(x)-g(x)}}{1+\abs{f(x)-g(x)}} } \\
		\tilde{d}(f,g) & \coloneqq  \sup_{x \in (0,1)} \frac{\abs{f(x)-g(x)}}{1+\abs{f(x)-g(x)}}
	\end{align*}
	to be two metrics on $C((0,1))$.
	Then $\tau_d \subset \tau_{\tilde{d}}$.
\end{proposition}
\begin{proof}
	First of all, for all $f,g \in C((0,1))$, for every $n \in \bbN^+$, we have
	\[
	\max_{x \in K_n} \frac{\abs{f(x)-g(x)}}{1+\abs{f(x)-g(x)}} \le \tilde{d}(f,g)
	\]
	and $\nicefrac{1}{2^n} < 1$ for all $n \in \bbN^+$, we know that $d(f,g) \le \tilde{d}(f,g)$.
	By \lemref{SmallerMetricIsWeaker} we know that $\tau_d \subseteq \tau_{\tilde{d}}$.
	To show $\tau_d \neq \tau_{\tilde{d}}$, we claim that $B_{\tilde{d}}(\mathbf{0},\nicefrac{1}{2})$ is not a neighborhood of $\mathbf{0}$ in $\tau_d$.
	Suppose it is, then there exists $r > 0$ such that $B_d(\mathbf{0},r) \subseteq B_{\tilde{d}}(\mathbf{0},\nicefrac{1}{2})$.
	Choose an $m >1$ such that $\nicefrac{1}{2^m} < r$.
	Define $f \in C((0,1))$ to be the piecewise linear functions through $(0,1)$, $(\nicefrac{1}{(m+1)},1)$, $(\nicefrac{1}{m},0)$, $(1-\nicefrac{1}{m},0)$, $(1-\nicefrac{1}{m+1},1)$ and $(1,1)$.
	Then for all $n \le m$, $f(K_n) = \{0\}$, so
	\[
	d(f,\mathbf{0})  = \max_{n > m} \set{ \frac{1}{2^n} \max_{x \in K_n} \frac{f(x)}{1+f(x)}  } < \frac{1}{2^m} < r
	\]
	however
	\[
	\tilde{d}(f,\mathbf{0}) = \sup_{x \in (0,1)} \frac{f(x)}{1+f(x)} = \frac{1}{2}
	\]
	thus $f \in B_d(\mathbf{0},r)$ and $f \not\in B_{\tilde{d}}(\mathbf{0}, \nicefrac{1}{2})$, which leads to a contradiction.
\end{proof}

\begin{definition}\label{De:QuotientMetricSpaces}
	Let $\rho$ be a pseudo-metric (i.e., a metric without the property as \defref{Metrics}(1)(b)) on $X$.
	Define an equivalence relation ${\sim}$ as $x \sim y$ iff $\rho(x,y) = 0$.
	Then $(Y,d)$ with $Y \coloneqq X / {\sim}$ and $d([x],[y]) \coloneqq \rho(x,y)$ is said to be the \emph{quotient metric space}.
\end{definition}

\begin{definition}\label{De:InifiniteSums}
	Let $X$ be a set and $f : X \to [0,+\infty]$.
	The \emph{infinite sum}, written as $\sum_{x \in X} f(x)$, is defined as
	\begin{equation*}
		\sup \set{ \sum_{x \in Y} f(x) : Y \subseteq X, Y~\text{is finite} }.
	\end{equation*}
\end{definition}

\begin{proposition}\label{Prop:FiniteSumImplyCountableNonZero}
	If $\sum_{x \in X} f(x) < +\infty$, then the set $\{ x \in X : f(x) > 0\}$ is countable.
	Moreover, $f$ does not take the value $+\infty$.
\end{proposition}
\begin{proof}
	Suppose $\sum_{x \in X} f(x) = M < +\infty$.
	For all $k \in \bbN^+$, define $X_k \coloneqq \{ x \in X : f(x) > \nicefrac{1}{k} \}$.
	Then
	\[
	\{ x \in X : f(x) > 0 \} = \bigcup_{k \in \bbN^+} X_k.
	\]
	On the other hand, we have
	\[
	\sum_{x \in X_k} f(x) \le M
	\]
	so indeed
	\[
	\abs{X_k} \le \frac{M}{\frac{1}{k}} = Mk
	\]
	is finite. Therefore we conclude that $\{x \in X : f(x) > 0\}$ is countable.
\end{proof}

\begin{definition}\label{De:Norms}
	Let $V$ be a vector space over $\bbR$.
	The function $\norm{\cdot} : V \to [0,+\infty)$ is said to be a \emph{norm} on $X$ if
	\begin{enumerate}
		\item $\norm{v} = 0$ iff $v = 0$ for all $v \in V$,
		\item $\norm{\alpha v} = \abs{\alpha} \cdot \norm{v}$ for all $v \in V$, $\alpha \in \bbR$, and
		\item $\norm{v+w} \le \norm{v} + \norm{w}$ for all $v,w \in V$.
	\end{enumerate}
	Then $(V,\norm{\cdot})$ is said to be a \emph{normed space}.
\end{definition}

\begin{proposition}\label{Prop:NormInduceMetric}
	If $(X,\norm{\cdot})$ is a normed space, then $d : X \times X \to [0,+\infty)$, defined as $(x,y) \mapsto \norm{x-y}$, is a metric on $X$.
\end{proposition}

\begin{remark}\label{Rem:NormedBalls}
	If $(X,\norm{\cdot})$ is a normed space, we will consider in $X$ the topology induced by the metric $d$ in \propref{NormInduceMetric}, i.e., the topology whose base of open sets are $\{ B(x,r) : x \in X, r > 0 \}$ with $B(x,r) \coloneqq \{ y \in X : \norm{x-y} < r \}$.
\end{remark}

\begin{remark}\label{Rem:MetricNotImplyNorm}
	In general, topological spaces don't imply metric spaces and metric spaces don't imply normed spaces.
\end{remark}

\begin{definition}\label{De:lpSpaces}
	Given a set $X$ and $p \in [1,+\infty)$, the space $\ell^p(X)$ is defined as
	\begin{equation*}
		\set{ f : X \to \bbR : \sum_{x \in X} \abs{f(x)}^p < +\infty }.
	\end{equation*}
\end{definition}

\begin{definition}\label{De:HoldersConjugateExponent}
	Suppose $1 \le p \le +\infty$.
	The \emph{H{\"o}lder's conjugate exponent} of $p$ is $p' \coloneqq \nicefrac{p}{(1-p)} \in [1,+\infty]$.
	Then $\nicefrac{1}{p} + \nicefrac{1}{p'}=1$ and $(p')' = p$.
\end{definition}

\begin{theorem}[Yang's inequality]\label{The:YangsInequality}
	Given $1 < p < +\infty$ and $a,b > 0$, we have $ab \le \nicefrac{1}{p}  \cdot a^p + \nicefrac{1}{p'} \cdot b^{p'}$.
\end{theorem}

\begin{theorem}[H{\"o}lder's inequality]\label{The:HoldersInequality}
	Let $X$ be a set.
	Given $1 \le p \le +\infty$, $q$ conjugate exponent of $p$, and $f,g: X \to [-\infty,+\infty]$, we have
	\begin{alignat*}{2}
		\sum_{x \in X} \abs{f(x) \cdot g(x)} & \le  \p{ \sum_{x \in X} \abs{f(x)}^p }^{\frac{1}{p}} \p{  \sum_{y \in X} \abs{g(y)}^q }^{\frac{1}{q}}, \quad &&  1 < p < +\infty \\
		\sum_{x \in X} \abs{f(x) \cdot g(x)} & \le \p{ \sum_{x \in X} \abs{f(x)} } \cdot \sup_{y \in X} \abs{g(y)}, \quad &&  p=1 \\
		\sum_{x \in X} \abs{f(x) \cdot g(x)} & \le \sup_{x \in X} \abs{f(x)} \cdot \p{ \sum_{y \in X} \abs{g(y)} }, \quad && p=+\infty
	\end{alignat*}
	As a corollary, if $f \in \ell^p(X)$ and $g \in \ell^q(X)$, then $f \cdot g \in \ell^1(X)$.
\end{theorem}

\begin{theorem}[Minkowski's inequality]\label{The:MinkowskisInequality}
	Let $X$ be a set.
	Given $1 \le p < +\infty$ and $f,g:X \to [-\infty,+\infty]$, we have
	\begin{equation*}
		\p{ \sum_{x \in X} \abs{f(x)+g(x)}^p }^{\frac{1}{p}} \le \p{ \sum_{x \in X} \abs{f(x)}^p }^{\frac{1}{p}} + \p{ \sum_{x \in X} \abs{g(x)}^p }^{\frac{1}{p}}.
	\end{equation*}
\end{theorem}

\begin{proposition}\label{Prop:lpVectorSpace}
	$\ell^p(X)$ is a vector space over $\bbR$ with the norm $\norm{f}_p \coloneqq  \p{\sum_{x \in X} \abs{f(x)}^p}^{\nicefrac{1}{p}}$.
\end{proposition}
\begin{proof}
	\begin{itemize}
		\item WTS \defref{Norms}(1):
		On the one hand, we have $\norm{\mathbf{0}}_p = 0$ trivially.
		On the other hand, if $\norm{f}_p = 0$ then we know that
		\[
		\sum_{x \in X} \abs{f(x)}^p = 0
		\]
		thus for all $x \in X$, we have $f(x) = 0$, i.e., $f = \mathbf{0}$.
		
		\item WTS \defref{Norms}(2): Suppose $f \in \ell^p(X)$ and $\alpha \in \bbR$.
		Then we have
		\begin{align*}
			\norm{\alpha f} & = \p{\sum_{x \in X} \abs{\alpha \cdot f(x)}^p}^{\frac{1}{p}} \\
			& = \p{ \sum_{x \in X} \abs{\alpha}^p \cdot \abs{f(x)}^p}^\frac{1}{p} \\
			& = \p{ \abs{\alpha}^p \cdot \sum_{x \in X} \abs{f(x)}^p }^\frac{1}{p} \\
			& = \abs{\alpha} \cdot \norm{f}_p.
		\end{align*}
		
		\item WTS \defref{Norms}(3): Suppose $f,g \in \ell^p(X)$ and we want to show that $\norm{f+g}_p \le \norm{f}_p+\norm{g}_p$, or equivalently
		\[
		\p{ \sum_{x \in X} \abs{f(x) + g(x)}^p }^\frac{1}{p} \le \p{ \sum_{x \in X} \abs{f(x)}^p}^\frac{1}{p} + \p{ \sum_{x \in X} \abs{g(x)}^p}^\frac{1}{p}.
		\]
		We then conclude this case by \theoref{MinkowskisInequality}.
	\end{itemize}
\end{proof}

\section{Limit Points}

\begin{definition}\label{De:LimitPoint}
	Let $(X,\tau)$ be a topological space.
	Given $E \subseteq X$ and $p \in X$, $p$ is said to be a \emph{limit point} of $E$ if for all neighborhoods $U$ of $p$, we have $(U \cap E) \setminus \{ p \} \neq \emptyset$.
\end{definition}

\begin{proposition}\label{Prop:LimitContainedInClosure}
	Let $E'$ be the set of all limit points of $E$.
	Then $E' \subseteq \overline{E}$.
\end{proposition}
\begin{proof}
	Let $p$ be a limit point of $E$, i.e., for all neighborhoods $U$ of $p$, $(U \cap E) \setminus \{ p\} \neq \emptyset$, so indeed $U \cap E \neq \emptyset$.
	By definition of closures we conclude that $p \in \overline{E}$.
\end{proof}

\begin{proposition}\label{Prop:NonLimitIsIsolated}
	Let $(X,\tau)$ be a topological space.
	Suppose $E \subseteq X$ is equipped with the induced topology $\tau_E$.
	Then $p \in E \setminus E'$ iff $\{p\} \in \tau_E$ (i.e., $p$ is an isolated point in $\tau_E$).
\end{proposition}
\begin{proof}
	\begin{align*}
		& \{p\} \in \tau_E \\
		\iff & \Exists{U \in \tau} U \cap E = \{p\} \\
		\iff & p \in E \wedge \Exists{U \in \tau} (U \cap E) \setminus \{p\} = \emptyset \\
		\iff & p \in E \wedge p \not\in E' \\
		\iff & p \in E \setminus E'.
	\end{align*}
\end{proof}

\begin{proposition}\label{Prop:BallOfLimitContainInfinite}
	Let $(X,d)$ be a metric space.
	Given $E \subseteq X$ and $p \in E'$, for all $r > 0$, the ball $B(p,r)$ contains infinitely many points of $E$.
\end{proposition}
\begin{proof}
	Suppose that there exists $r > 0$ such that $B(p,r)$ contains finitely many points of $E$.
	Then there exists $r' > 0$ such that $B(p,r') = \{p\}$.
	But $B(p,r')$ is an open neighborhood of $p$, $p \in E'$ implies that $(B(p,r') \cap E) \setminus \{p\}$ should be nonempty.
	Thus we conclude the proof by contradiction.
\end{proof}

\begin{proposition}\label{Prop:OrigPlusLimitIsClosure}
	Let $(X,\tau)$ be a topological space and $E \subseteq X$.
	Then $E \cup E' = \overline{E}$.
\end{proposition}
\begin{proof}
	By the fact that $E \subseteq \overline{E}$ as well as \propref{LimitContainedInClosure} we know that $E \cup \overline{E} \subseteq \overline{E}$.
	Then it suffices to show that $\overline{E} \subseteq E \cup E'$.
	Suppose $p \in \overline{E}$, then for every neighborhood $U$ of $p$, $U \cap E \neq \emptyset$.
	\begin{itemize}
		\item If for all such $U$, $(U \cap E) \setminus \{p\} \neq \emptyset$, then $p \in E'$.
		\item Otherwise, suppose there exists $U_0 \in \tau$ such that $p \in U_0$ and $(U_0 \cap E) \setminus \{p\} = \emptyset$. Because $U_0 \cap E \neq \emptyset$, we know that $U_0 \cap E = \{ p\}$ exactly. Thus $p \in E$.
	\end{itemize}
\end{proof}

\section{Sequences}

\begin{definition}\label{De:Sequences}
	Let $(X,\tau)$ be a topological space.
	Suppose $\{x_n\}_{n \in \bbN} \subseteq X$ is a sequence in $X$.
	Then $p$ is said to be a \emph{limit} of $\{x_n\}_{n \in \bbN}$, written as $x_n \xrightarrow[n \to \infty]{X} p$ (sometimes $x_n \xrightarrow{X} p$ or even $x_n \rightarrow p$ if there is no ambiguity) , if for all neighborhoods $U$ of $p$, there exists $n_0 \in \bbN$ such that for all $n \ge n_0$, we have $x_n \in U$.
\end{definition}

\begin{remark}\label{Rem:LimitOfSeqNotUnique}
	The limit of a sequence may not be unique.
\end{remark}

\begin{lemma}\label{Lem:LimitOfSeqInClosure}
	Let $(X,\tau)$ be a topological space and $E \subseteq X$.
	If $x_n \rightarrow p$ for some sequence $\{x_n\}_{n \in \bbN} \subseteq E$ and $p \in X$, then $p \in \overline{E}$.
\end{lemma}
\begin{proof}
	Straightforward from the definition.
\end{proof}

\begin{lemma}\label{Lem:FirstCountLimitOfSeqIffClosure}
	If a topological space satisfies the first axiom of countability, then $p \in \overline{E}$ iff there exists a sequence $\{x_n\}_{n \in \bbN} \subseteq E$ such that $x_n \rightarrow p$.
\end{lemma}
\begin{proof}
	The ${\impliedby}$ direction is given by \lemref{LimitOfSeqInClosure}.
	For the ${\implies}$ direction, suppose $\beta_p = \{B_n\}_{n \in \bbN}$ is a local base for $p$.
	\begin{itemize}
		\item If $p \in E$, then we define $x_n \coloneqq p$ for all $n$.
		Then $\{x_n\}_{n \in \bbN} \subseteq E$ and $x_n \rightarrow p$ trivially.
		\item If $|\beta_p|$ is finite, i.e., $\beta_p = \{B_1,\cdots,B_N\}$ for some $N$, we define $V \coloneqq B_1 \cap \cdots \cap B_N$, then $V$ is open and by the fact that $p \in V$ we know that $V$ is a neighborhood of $p$.
			By the assumption that $p \in \overline{E}$ we know that $V \cap E \neq \emptyset$.
			Pick $x_o \in V \cap E$ and define $x_n \coloneqq x_o$ for all $n$.
			Then $\{x_n\}_{n \in \bbN} \subseteq E$ and $x_n \rightarrow p$ trivially.
		\item Otherwise, suppose $\beta=\{B_n\}_{n \in \bbN}$ contains countably many elements.
			By $p \in \overline{E}$ we know that $E \cap B_n \neq \emptyset$ for all $n$.
			We define
			\[
			x_n \in \p{ \bigcap_{i=0}^n B_i } \cap E
			\]
			and we claim that $x_n \rightarrow p$.
			Suppose $U$ is a neighborhood of $p$.
			Then there exists $m \in \bbN$ such that $U \supseteq B_m$.
			By definition of $\{x_n\}_{n \in \bbN}$, we conclude that for all $n \ge m$, $x_n \in B_m \subseteq U$.
	\end{itemize}
\end{proof}

\begin{example}\label{Exa:CountableNotImplyFstCountable}
	In general countable sets do not imply the first axiom of countability.
	Define $X \coloneqq \bbN \times \bbN$.
	Define $\tau$ as follows: $U \subseteq X$ is open if (i) either $(0,0) \not\in U$, or (ii) $(0,0) \in U$ implies that $U$ contains all but a finite number of points in all but a finite number of columns.
	Then $\tau$ is not first countable.
\end{example}

\begin{definition}\label{De:SequentiallyClosed}
	Let $(X,\tau)$ be a topological space.
	Then $E \subseteq X$ is said to be \emph{sequentially closed} if $E = \{ p \in X : \Exists{\{x_n\}_{n \in \bbN} \subseteq E } x_n \rightarrow p \}$.
\end{definition}

\begin{definition}\label{De:SequentiallyOpen}
	Let $(X,\tau)$ be a topological space.
	Then $U \subseteq X$ is said to be \emph{sequentially open} if for all $p \in U$, for all $\{x_n\}_{n \in \bbN} \subseteq X$ with $x_n \rightarrow p$, there exists $n_0 \in \bbN$ such that for all $n \ge n_0$, we have $x_n \in U$.
\end{definition}

\begin{proposition}\label{Prop:ClosedOpenImplySeqClosedOpen}\
	\begin{itemize}
		\item If $E$ is closed, then $E$ is sequentially closed.
		\item If $U$ is open, then $U$ is sequentially open.
	\end{itemize}
\end{proposition}
\begin{proof}
	\begin{itemize}
		\item If $E$ is closed, then we have $E = \overline{E}$.
		By \lemref{LimitOfSeqInClosure} we know that for all sequences $\{x_n\}_{n \in \bbN} \subseteq E$ and $x_n \rightarrow p$, then $p \in \overline{E}$.
		Therefore $E$ is sequentially closed.
		
		\item Suppose $U$ is open and $\{x_n\}_{n \in \bbN} \subseteq X$ with $x_n \rightarrow p$ for some $p \in U$.
		We claim that there exists $n_0 \in \bbN$ such that for all $n \ge n_0$ we have $x_n \in U$.
		If not, we know that for all $n_0 \in \bbN$ there exists $n \ge n_0$ such that $x_n \not\in U$.
		Thus the sequence has a sub-sequence $\{y_n\}_{n \in \bbN} \subseteq X \setminus U$ with the same limit point $p$.
		However, $X \setminus U$ is closed and hence sequentially closed, we have $p \in X \setminus U$, which contradicts the assumption that $p \in U$.
	\end{itemize}
\end{proof}

\begin{proposition}\label{Prop:SeqOpenNotOpen}
	In general, sequentially open sets may not be open.
	Consider an uncountable set $X$ and the \emph{co-countable topology} $\tau \coloneqq \{ X \setminus F : F~\text{is countable} \} \cup \{ \emptyset \}$.
	Then for all $p \in X$, $\{p\}$ is not open, while all nonempty sets $U \subseteq X$ are sequentially open.
\end{proposition}
\begin{proof}
	\begin{itemize}
		\item WTS $\{p\}$ is not open for all $p \in X$: By definition of $\tau$ we know that all nonempty open sets should contain uncountably many elements. Thus $\{p\}$ is not open.
		\item WTS $E$ is sequentially open for all nonempty $E \subseteq X$: Suppose $p \in E$ and $\{x_n\}_{n \in \bbN} \subseteq X$ with $x_n \rightarrow p$.
		We want to show that there exists $n_0 \in \bbN$ such that for all $n \ge n_0$ we have $x_n \in E$.
		Suppose not, then for all $n_0 \in \bbN$, there exists $n \ge n_0$ such that $x_{n_0} \not\in E$.
		Then there exists a sub-sequence $\{ y_n\}_{n \in \bbN} \subseteq X \setminus E$ with the same limit point $p$.
		Define $U \coloneqq X \setminus \{y_n : n \in \bbN\}$.
		Then $U$ is an open neighborhood of $E$, and indeed a neighborhood of $p$.
		However, since $y_n \not\in U$ for all $n$, we have a contradiction from $y_n \rightarrow p$.
	\end{itemize}
\end{proof}

\begin{remark}\label{Rem:ClosureIsSeqClosureThenFUSpace}
	Define $\overline{E}_{\mathrm{seq}} \coloneqq \{ p \in X : \Exists{\{x_n\}_{n \in \bbN} \subseteq E} x_n \rightarrow p \}$.
	Then in general we have $\overline{E}_{\mathrm{seq}} \subseteq \overline{E}$.
	If $(X,\tau)$ is a topological space where for all $E \subseteq X$ we have $\overline{E} = \overline{E}_{\mathrm{seq}}$, then $(X,\tau)$ is said to be a \emph{Fr{\'e}chet-Urysohn space}.
	The first axiom of countability indeed implies F-U.
\end{remark}

\begin{definition}\label{De:SequentialSpace}
	Let $(X,\tau)$ be a topological space.
	If every sequentially closed subset $E \subseteq X$ is closed and every sequentially open subset $U \subseteq X$ is open, then $(X,\tau)$ is said to be a \emph{sequential space}.
\end{definition}
	
\section{Separability}

\begin{definition}\label{De:DenseSet}
	Let $(X,\tau)$ be a topological space.
	$E$ is said to be \emph{dense} if $\overline{E} = X$.
\end{definition}

\begin{definition}\label{De:SeparableSpace}
	A topological space $(X,\tau)$ is said to be \emph{separable} if there exists a countable dense set in $X$.
\end{definition}

\begin{proposition}\label{Prop:SubsetSeparable}
	If $\tau_1 \subseteq \tau_2$ and $(X,\tau_2)$ is separable, then $(X,\tau_1)$ is also separable.
\end{proposition}
\begin{proof}
	Let $E$ be a dense set in $\tau_2$.
	We claim that $E$ is also dense in $\tau_1$.
	Let $p \in X$ and $U \in \tau_1$ such that $p \in U$.
	By the assumption that $\tau_1 \subseteq \tau_2$ we know that $U \in \tau_2$.
	Because $E$ is dense in $\tau_2$, we know that $U \cap E \neq \emptyset$.
	Then by definition we know that $p \in \overline{E}^{\tau_1}$, and we conclude that $E$ is dense in $\tau_1$.
\end{proof}

\begin{proposition}\label{Prop:SndAxCountImplySeparable}
	If a topological space $(X,\tau)$ satisfies the second axiom of countability, then $(X,\tau)$ is separable.
\end{proposition}
\begin{proof}
	Let $\beta = \{ B_n\}_{n \in \bbN}$ is a base for $\tau$.
	Define $x_n \in B_n$ for all $n \in \bbN$ and $E \coloneqq \{ x_n : n \in \bbN \}$.
	We claim that $\overline{E} = X$.
	For all $p \in X$, let $U$ be a neighborhood of $p$.
	Then there exists a collection $\{B_n\}_{n \in \calI}$ with some $\calI \subseteq \bbN$ such that
	\[
	U = \bigcup_{n \in \calI} B_n.
	\]
	Therefore $U \cap E \supseteq \{ x_n : n \in \calI \} \neq \emptyset$.
	By definition of closures we know that $p \in \overline{E}$.
	Therefore we conclude that $X = \overline{E}$.
\end{proof}

\begin{proposition}\label{Prop:MetricSeparableImplySndAxCount}
	Suppose $(X,d)$ is a metric space.
	Then $(X,d)$ is separable iff it is second countable.
\end{proposition}
\begin{proof}
	The ${\impliedby}$ direction is proved in \propref{SndAxCountImplySeparable}.
	To show the ${\implies}$ direction, let $E$ be a countable dense set in $\tau_d$.
	Define a countable collection of open sets
	\[
	\beta \coloneqq \set{ B(x,r) : x \in E, r \in \bbQ^+}
	\]
	and we claim that $\beta$ is a base for $\tau_d$.
	Let $U \in \tau_d$ be an open set.
	Then for each $x \in U$, there exists $r_x > 0$ such that $B(x,r_x) \subseteq U$.
	On the other hand, $\overline{E} = X$ implies that $B(x,\nicefrac{r_x}{2}) \cap E \neq \emptyset$, thus we can pick $e_x \in B(x,\nicefrac{r_x}{2}) \cap E$.
	Then $x \in B(e_x,\nicefrac{r_x}{2})$.
	On the other hand, for every $y \in B(e_x,\nicefrac{r_x}{2})$, we have
	\[
	d(y,x) \le d(y,e_x) + d(e_x,x) < \frac{r_x}{2} + \frac{r_x}{2} = r_x
	\]
	thus $y \in B(x,r_x)$.
	Hence we know that $B(e_x,\nicefrac{r_x}{2}) \subseteq B(x,r_x) \subseteq U$.
	Therefore, we conclude the proof by
	\[
	U = \bigcup_{x \in U} B\p{e_x, \frac{r_x}{2}}.
	\]
\end{proof}

\begin{example}\label{Exa:SeparableSpaces}\
	\begin{itemize}
		\item $\bbR$ with the Sorgenfrey topology is separable. Indeed, we have $\overline{\bbQ} = \bbR$.
		\item Suppose $X \neq \emptyset$ is equipped with the discrete topology.
		Then $X$ is separable iff $X$ is countable.
		\item $\ell^p(\bbN)$ for some $1 \le p < +\infty$ is separable.
		A countable dense set could be $\{  \{ (r_1,\cdots,r_n,0,\cdots) \in \ell^p(\bbN) : n \in \bbN, r_i \in \bbQ \}$.
		\item $C([a,b])$ with the norm $\norm{f}_\infty \coloneqq \max_{x \in [a,b]} \abs{f(x)}$ is separable.
	\end{itemize}
\end{example}

\begin{proposition}\label{Prop:SeparableMetricInduceSeparable}
	Suppose $(X,d)$ is a separable metric space and $A \subseteq X$.
	Then $(A,d|_A)$ is also separable.
\end{proposition}
\begin{proof}
	By \propref{MetricSeparableImplySndAxCount} we know that $(X,d)$ satisfies the second axiom of countability.
	By \remref{AxCountHereditary} we know that $(A,d|_A)$ is also second countable.
	Again by \propref{MetricSeparableImplySndAxCount} we conclude that $(A,d|_A)$ is separable.
\end{proof}

\section{Connectedness}

\begin{definition}\label{De:ConnectedSpace}
	A topological space $(X,\tau)$ is said to be \emph{connected} if the only open sets that are also closed are $\emptyset$ and $X$.
\end{definition}

\begin{remark}\label{Rem:SplitDisconnectedSpace}
	$(X,\tau)$ said to be is \emph{disconnected} if there exists $A \subseteq X$ such that $A \neq \emptyset$, $A \neq X$ and $A$ is both open and closed.
	In other words, $X = A \cup (X \setminus A)$ is a union of two disjoint open sets.
	In general, $(X,\tau)$ is disconnected if there exist nonempty open sets $U_1$ and $U_2$ such that $U_1 \cap U_2 = \emptyset$ and $X = U_1 \cup U_2$.
\end{remark}

\begin{definition}\label{De:ConnectedSet}
	Let $(X,\tau)$ be a topological space.
	$E \subseteq X$ is said to be \emph{connected} if the induced topology $(E,\tau_E)$ is connected.
\end{definition}

\begin{remark}\label{Rem:DisconnectedSubsetNotLifting}
	Let $(X,\tau)$ be a topological space.
	If $E \subseteq X$ is disconnect, then $E = V_1 \cup V_2$ for some disjoint nonempty open sets $V_1,V_2$ in $E$.
	Then there exist open sets $U_1$ and $U_2$ in $X$ such that $V_1 = E \cap U_1$ and $V_2 = E \cap U_2$.
	However, in general $U_1 \cap U_2 \neq \emptyset$.
\end{remark}

\begin{proposition}\label{Prop:MetricDisconnectedSubsetLifting}
	Suppose $(X,d)$ is a metric space.
	Then $E \subseteq X$ is disconnected iff there exist nonempty disjoint open sets $U$ and $V$ in $X$ such that $U \cap E \neq \emptyset$, $V \cap E \neq \emptyset$, and $E \subseteq U \cup V$.
\end{proposition}
\begin{proof}
	\begin{itemize}
		\item WTS the ${\impliedby}$ direction: Observe that
		\[
		E = (U \cap E) \cup (V \cap E)
		\]
		by $E \subseteq U \cup V$.
		By $U$ and $V$ are disjoint we know that $U \cap E$ and $V \cap E$ are also disjoint.
		Also, we know that $U \cap E$ and $V \cap E$ are nonempty.
		Thus by definition we conclude that $E$ is disconnected.
		
		\item WTS the ${\implies}$ direction:
		Suppose $E \subseteq X$ is disconnected.
		Then there exists sets $A,B$ such that
		\begin{enumerate}[(i)]
			\item $E = A \cup B$,
			\item $A = E \cap U$ for some $U \in \tau_d$,
			\item $B = E \cap V$ for some $V \in \tau_d$,
			\item $A \neq \emptyset$, $B \neq \emptyset$, and
			\item $A \cap B = \emptyset$ and thus $E \cap U \cap V = \emptyset$.
		\end{enumerate}
		
		First, we claim that $A \cup \overline{B} = \emptyset$ and $\overline{A} \cup B = \emptyset$.
		Without loss of generality we only need to show $\overline{A} \cup B = \emptyset$.
		Let $b \in B = E \cap V$.
		Then there exists $r > 0$ such that $B(b,r) \subseteq V$.
		Thus we have
		\[
		B(b,r) \cap A \subseteq V \cap A = V \cap E \cap U = \emptyset
		\]
		by (v) and we know that $b \not\in \overline{A}$.
		
		By $A \cup \overline{B} = \emptyset$, we know that for every $a \in A$, there exists $r_a > 0$ such that $B(a,r_a) \cap B = \emptyset$.
		Similarly, for every $b \in B$, there exists $r_b > 0$ such that $B(b,r_b) \cap A = \emptyset$.
		Define two open sets
		\begin{align*}
			\tilde{U} & \coloneqq \bigcup_{a \in A} B\p{a,\frac{r_a}{2}} \\
			\tilde{V} & \coloneqq \bigcup_{b \in B} B\p{b,\frac{r_b}{2}}
		\end{align*}
		and then we have
		\begin{itemize}
			\item $\tilde{U} \cap E \supseteq A \cap E  = A \neq \emptyset$ by (i) and (iv),
			\item $\tilde{V} \cap E \supseteq B \cap E = B \neq \emptyset$ by (i) and (iv),
			\item $E = A \cup B \subseteq \tilde{U} \cap \tilde{V}$ by (i), and
			\item $\tilde{U} \cap \tilde{V} = \emptyset$, because for all $x \in B(a,\nicefrac{r_a}{2})$ and $y \in B(b,\nicefrac{r_b}{2})$, we know the following
			\begin{align*}
				& d(a,b) \le d(a,x) + d(x,y) + d(y,b) \\
				\implies & d(x,y) \ge d(a,b) - d(a,x) - d(y,b) \\
				\implies & d(x,y) > \max\{r_a,r_b\} - \frac{r_a}{2} - \frac{r_b}{2} \\
				\implies & d(x,y) > 0 \\
				\implies & x \neq y
			\end{align*}
			hence $\tilde{U} \cap \tilde{V} = \emptyset$.
		\end{itemize}
	\end{itemize}
\end{proof}

\begin{theorem}\label{The:RConnectedConvex}
	$C \subseteq \bbR$ is connected iff $C$ is convex, i.e., an interval.
\end{theorem}
\begin{proof}
	\diw{TODO.}
\end{proof}

\begin{proposition}\label{Prop:ConnIffEveryTwoPointsConn}
	A topological space $(X,\tau)$ is connected iff for all $x,y\in X$, there exists $E \subseteq X$ such that $x,y\in E$ and $E$ is connected.
\end{proposition}
\begin{proof}
	\begin{itemize}
		\item WTS the ${\implies}$ direction: Set $E$ to be $X$.
		\item WTS the ${\impliedby}$ direction:
		Suppose $X$ is disconnected.
		Then there exist disjoint nonempty open sets $\tilde{U}$ and $\tilde{V}$ such that $\tilde{U} \cup \tilde{V} = X$.
		Let $x \in \tilde{U}$ and $y \in \tilde{V}$.
		Then by assumption we know that there exists $E \subseteq X$ such that $x,y\in E$ and $E$ is connected.
		However, on the other hand, we have
		\[
		E = E \cap X = (E \cap \tilde{U}) \cup (E \cap \tilde{V})
		\]
		and indeed $E$ is union of two disjoint nonempty open sets in $E$, thus $E$ should be disconnected.
		By contradiction, we conclude that $X$ is connected.
	\end{itemize}
\end{proof}

\begin{proposition}\label{Prop:UnionConnStillConn}
	Let $(X,\tau)$ be a topological space.
	Let $E \subseteq X$.
	If $E = \bigcup_{\alpha \in \Lambda} E_\alpha$ for each $E_\alpha$ connected, and $\bigcap_{\alpha \in \Lambda} E_\alpha$ is nonempty, then $E$ is connected.
\end{proposition}
\begin{proof}
	Without loss of generality let's assume $E = X$.
	Suppose $E$ is disconnected.
	Then there exist disjoint nonempty open sets $U,V$ in $E$ such that $E = U \cup V$.
	On the other hand, we have for each $\alpha \in \Lambda$
	\[
	E_\alpha = E \cap E_\alpha = (U \cap E_\alpha) \cup (V \cap E_\alpha),
	\]
	but $E_\alpha$ is connected, hence either $E_\alpha \subseteq U$ or $E_\alpha \subseteq V$.
	By the assumption that $\bigcap_{\alpha \in \Lambda} E_\alpha \neq \emptyset$, we know that either for all $\alpha$, $E_\alpha \subseteq U$ or for all $\alpha$, $E_\alpha \subseteq V$.
	Because $E = \bigcup_{\alpha \in \Lambda} E_\alpha$, we know that either $E \subseteq U$ or $E \subseteq V$, thus either $U = \emptyset$ or $V = \emptyset$, which leads to a contradiction.
	Therefore we conclude that $E$ is connected.
\end{proof}

\begin{proposition}\label{Prop:ClosureOfConnStillConn}
	Let $(X,\tau)$ be a topological space.
	If $E \subseteq X$ is connected, then $\overline{E}$ is also connected.
\end{proposition}
\begin{proof}
	Suppose $\overline{E}$ is disconnected.
	Then there exist two disjoint nonempty open sets $U,V$ in $\overline{E}$ such that $\overline{E} = U \cup V$.
	Let $U = \overline{E} \cap U_0$ and $V = \overline{E}\cap V_0$ for some $U_0,V_0 \in \tau$.
	Thus by $E \subseteq \overline{E}$ we have
	\[
	E = (E \cap U_0) \cup (E \cap V_0).
	\]
	On the other hand,
	\[
	(E \cap U_0) \cap (E \cap V_0) \subseteq (\overline{E} \cap U_0) \cap (\overline{E} \cap V_0) = U \cap V = \emptyset,
	\]
	and because $E$ is connected, we have either $E \subseteq U_0$ or $E \subseteq V_0$.
	Without loss of generality, let's assume $E \subseteq V_0$ and indeed $E \cap U_0 = \emptyset$.
	However, $\overline{E} \cap U_0 = U \neq \emptyset$, so pick an $x \in \overline{E} \cap U_0$, by $x \in \overline{E}$ and $x \in U_0$ we know that $U_0 \cap E \neq \emptyset$, which leads to a contradiction.
	Therefore we conclude that $\overline{E}$ is connected.
\end{proof}

\begin{definition}\label{De:ContinuousFromInterval}
	Let $(X,\tau)$ be a topological space and $[a,b] \subseteq \bbR$.
	A function $r : [a,b] \to X$ is said to be \emph{continuous} if for all $x_0 \in [a,b]$, for all sequences $\{x_n\}_{n \in \bbN} \subseteq [a,b]$, $x_n \xrightarrow{[a,b]} x_0$ implies $r(x_n) \xrightarrow{X} r(x_0)$.
\end{definition}

\begin{lemma}\label{Lem:ContinuousReverseOpenIsOpen}
	Let $(X,\tau)$ be a topological space and $[a,b] \subseteq \bbR$.
	Let $r : [a,b] \to X$ be a continuous function.
	Then for all $U \in \tau$, $r^{-1}(U)$ is open in $[a,b]$.
\end{lemma}
\begin{proof}
	Suppose $U \in \tau$ and $r^{-1}(U)$ is not open in $[a,b]$.
	Then there exists $x \in r^{-1}(U)$, such that for all $\epsilon > 0$, $(x-\epsilon,x+\epsilon) \cap [a,b] \not\subseteq r^{-1}(U)$.
	Define a sequence $\{x_n\}_{n \in \bbN^+}$ such that for all $n \in \bbN^+$, $x_n \in ((x-\nicefrac{1}{n},x+\nicefrac{1}{n}) \cap [a,b]) \setminus r^{-1}(U)$.
	Thus $r(x_n) \not\in U$ for all $n \in \bbN^+$.
	Then $x_n \xrightarrow{[a,b]} x$ and because $r$ is continuous, we know that $r(x_n) \xrightarrow{X} r(x) \in U$.
	By contradiction, we conclude that $r^{-1}(U)$ must be open.
\end{proof}

\begin{proposition}\label{Prop:ContinuousReverseClosedIsClosed}
	Let $r : [a,b] \to X$ be a continuous function.
	If $C$ is closed in $X$, then $r^{-1}(C)$ is closed in $[a,b]$.
\end{proposition}
\begin{proof}
	By \lemref{ContinuousReverseOpenIsOpen}, we know that $r^{-1}(X \setminus C)$ is open.
	Then
	\[
	r^{-1}(C) = r^{-1}(X \setminus (X \setminus C)) = r^{-1}(X) \setminus r^{-1}(X \setminus C) = [a,b] \setminus r^{-1}(X\setminus C)
	\]
	is closed in $[a,b]$.
\end{proof}

\begin{definition}\label{De:PathwiseConnected}
	A topological space $(X,\tau)$ is said to be \emph{pathwise connected} if for all $x,y \in X$, there exists a continuous path $r : [a,b] \to X$ such that $r(a) =x$ and $r(b)=y$.
\end{definition}

\begin{proposition}\label{Prop:PathWiseConnImplyConn}
	If the topological space $(X,\tau)$ is pathwise connected, then it is connected.
\end{proposition}
\begin{proof}
	Suppose $(X,\tau)$ is pathwise connected but not connected.
	Then there exist two disjoint nonempty open sets $U,V$ such that $X = U \cup V$.
	Let $x \in U$ and $y \in V$.
	By the pathwise connectedness, there exists a continuous path $r : [0,1] \to X$ such that $r(0)=x$ and $r(1)=y$.
	By \lemref{ContinuousReverseOpenIsOpen} and the fact that $r(0)=x,r(1)=y$, we know that $r^{-1}(U)$ and $r^{-1}(V)$ are nonempty open sets in $[a,b]$.
	By $U \cap V = \emptyset$, we know that $r^{-1}(U) \cap r^{-1}(V) = \emptyset$.
	On the other hand,
	\[
	[0,1] = r^{-1}(X) = r^{-1}(U) \cup r^{-1}(V)
	\]
	is written as union of two disjoint nonempty open sets, which contradicts the fact that $[0,1]$ is connected (by \theoref{RConnectedConvex}).
	Therefore we conclude that $(X,\tau)$ is connected.
\end{proof}

\begin{proposition}\label{Prop:OpenSetInEuclideanSpaceConn}
	Suppose $U \subseteq \bbR^N$ is an open set.
	Then $U$ is connected iff it is pathwise connected.
\end{proposition}
\begin{proof}
	\diw{TODO.}
\end{proof}

\begin{definition}\label{De:ConnEquivalenceRelation}
	Let $(X,\tau)$ be a topological space.
	Let's define a relation ${\sim}$ as $x \sim y$ iff there exists $E \subseteq X$ such that $x,y\in E$ and $E$ is connected.
	Then ${\sim}$ is an equivalence relation.
	For all $x \in X$, define the equivalence class of $x$ as $E_x \coloneqq \{ y \in X : y \sim x \}$.
	Then $E_x = \bigcup_{E \subseteq X, x \in E, E~\text{is connected}} E$, i.e., $E_x$ is a (largest) connected component containing $x$.
	Moreover, we have $E_x = \overline{E_x}$ by \propref{ClosureOfConnStillConn}.
\end{definition}

\begin{proposition}\label{Prop:ClosedConnComponentStillClosed}
	Let $(X,\tau)$ be a topological space and $C \subseteq X$ be a closed set.
	The connected components of $C$ are closed in $X$.
\end{proposition}

\begin{definition}\label{De:TotalDisconnectedness}
	A topological space $(X,\tau)$ is said to be \emph{totally disconnected} if every connected component is a singleton.
\end{definition}

\begin{example}\label{De:QInRIsTotallyDisconn}
	$\bbQ$ in $\bbR$ is totally disconnected.
\end{example}

\section{Functions and Continuity}

\begin{definition}\label{De:ContinuousFunctions}
	Let $(X,\tau_X)$ and $(Y,\tau_Y)$ be two topological spaces.
	A function $f : X \to Y$ is said to be \emph{continuous} if for all open sets $V \subseteq Y$ (i.e., $V \in \tau_Y$), we have $f^{-1}(V) \in \tau_X$.
\end{definition}

\begin{proposition}\label{Prop:SomeEquivResultsAboutContFunc}
	Let $(X,\tau_X)$ and $(Y,\tau_Y)$ be two topological spaces.
	Suppose $f : X \to Y$.
	The following are equivalent:
	\begin{enumerate}
		\item $f$ is continuous,
		\item for all $x \in X$, for all neighborhoods $V$ of $f(x)$ in $Y$, there exists a neighborhood $U$ of $x$ in $X$ such that $f(U) \subseteq V$, and
		\item $C$ is a closed set in $Y$ implies $f^{-1}(C)$ is closed in $X$.
	\end{enumerate}
\end{proposition}
\begin{proof}
	\begin{itemize}
		\item WTS (1) implies (2): Suppose $x \in X$ and $V$ is a neighborhood of $f(x)$ in $Y$.
		Because $f$ is continuous, we know that $f^{-1}(V) \in \tau_X$.
		Define $U \coloneqq f^{-1}(V)$ and then immediately we have $x \in U$ and $f(U) \subseteq V$.
		
		\item WTS (3) implies (1):
		For each $V \in \tau_Y$, defined $C \coloneqq Y \setminus V$ is a closed set in $Y$.
		Then $f^{-1}(C)$ is closed in $X$.
		Thus
		\[
		f^{-1}(V) = X \setminus f^{-1}(C)
		\]
		is open in $X$.
		
		\item WTS (2) implies (3): Let $C$ be a closed set in $Y$.
		Then $V \coloneqq Y \setminus C$ is an open set in $Y$.
		Define $U \coloneqq f^{-1}(V)$.
		If $U = \emptyset$, then $f^{-1}(C) = X$ is indeed closed in $X$.
		Suppose $U \neq \emptyset$.
		Then for every $x \in U$, there exists a neighborhood $U_x$ of $x$ such that $f(U_x) \subseteq V$.
		Therefore $U_x \subseteq f^{-1}(V) = U$ and
		\[
		U = \bigcup_{x \in U} U_x
		\]
		is indeed an open set in $X$.
		Therefore $f^{-1}(C) = X \setminus f^{-1}(V) = X \setminus U$ is closed in $X$.
	\end{itemize}
\end{proof}

\begin{proposition}\label{Prop:ContFuncIffOnSubbase}
	Let $(X,\tau_X)$ and $(Y,\tau_Y)$ be two topological spaces.
	Let $\calS$ be a subbase of open sets in $\tau_Y$.
	Then $f : X \to Y$ is continuous iff for all $V \in \calS$, $f^{-1}(V) \in \tau_X$.
\end{proposition}
\begin{proof}
	\diw{TODO.}
\end{proof}

\begin{definition}\label{De:SequentiallyContinuous}
	Let $(X,\tau_X)$ and $(Y,\tau_Y)$ be two topological spaces.
	The function $f : X \to Y$ is said to be \emph{sequentially continuous} if for all $x_0 \in X$, for all sequences $\{x_n\}_{n \in \bbN^+} \subseteq X$, $x_n \xrightarrow{X} x_0$ implies $f(x_n) \xrightarrow{Y} f(x_0)$.
\end{definition}

\begin{proposition}\label{Prop:ContImplySeqCont}
	If $f$ is a continuous function, then it is sequentially continuous.
\end{proposition}
\begin{proof}
	Let $f : X \to Y$ from $(X,\tau_X)$ to $(Y,\tau_Y)$.
	Suppose $x_n \xrightarrow{X} x_0$.
	We claim that $f(x_n) \xrightarrow{Y} f(x_0)$.
	Let $V$ be a neighborhood of $f(x_0)$ in $Y$.
	Because $f$ is continuous, we know that $U \coloneqq f^{-1}(V)$ is open in $X$.
	By $x_n \xrightarrow{X} x_0$ and $x_0 \in U$ we know that there exists $n_0$ such that for all $n \ge n_0$ we have $x_n \in U$, which implies that $f(x_n) \in V$.
	Therefore we conclude that $f(x_n) \xrightarrow{Y} f(x_0)$.
\end{proof}

\begin{example}\label{Exa:SeqContNotCont}
	Let $X = \bbR$ and $\tau_c$ be the co-countable topology $\{ \emptyset \} \cup \{ \bbR \setminus E : E~\text{is countable} \}$.
	Let $Y = X$ equipped with the discrete topology $\tau_d = \wp(\bbR)$.
	Let $f : X \to Y$ be a function defined as $x \mapsto x$.
	Then $f$ is sequentially continuous but not continuous.
\end{example}

\begin{proposition}\label{Prop:FstAxCountContIffSeqCont}
	Let $(X,\tau_X)$ and $(Y,\tau_Y)$ be two topological spaces.
	Let $f : X\to Y$.
	If $X$ satisfies the first axiom of countability, then $f$ is continuous iff $f$ is sequentially continuous.
\end{proposition}
\begin{proof}
	We already proved the ${\implies}$ direction in \propref{ContImplySeqCont}.
	To show the ${\impliedby}$ direction, suppose that $f$ is sequentially continuous but not continuous.
	Then there exists $V \in \tau_Y$ such that $f^{-1}(V) \not\in \tau_X$.
	In particular, $f^{-1}(V) \neq \emptyset$ so there exists $a \in f^{-1}(V)$ which is not an interior point of $f^{-1}(V)$.
	Let $\beta_a = \{B_n\}_{n\in \bbN}$ be a local base of $a$.
	Without loss of generality, let's assume $B_n \supseteq B_{n+1}$ for all $n \in \bbN$.
	Because $a$ is not an interior point of $f^{-1}(V)$, we know that for all $n \in \bbN$, $B_n \not\subseteq f^{-1}(V)$.
	Define a sequence $\{x_n\}_{n \in \bbN}$ such that $x_n \in B_n \setminus f^{-1}(V)$ for all $n$.
	Then we have $x_n \xrightarrow{X} a$.
	Because $f$ is sequentially continuous, we know that $f(x_n) \xrightarrow{Y} f(a) \in V$.
	However $f(x_n) \not\in V$ for all $n$, which leads to a contradiction.
	Therefore we conclude that $f$ is continuous.
\end{proof}

\begin{proposition}\label{Prop:ContCompIsCont}
	The composition of continuous functions is continuous.
\end{proposition}

\begin{proposition}\label{Prop:ContPreserveConn}
	Let $(X,\tau_X)$ and $(Y,\tau_Y)$ be two topological spaces.
	If $f : X \to Y$ is a continuous function, then $X$ is connected implies that $f(X)$ is connected.
\end{proposition}
\begin{proof}
	Suppose that $X$ is connected but $f(X)$ is not connected.
	Then there exist disjoint nonempty open sets $U,V$ in $Y$ such that $U \cap f(X) \neq \emptyset$, $V \cap f(X) \neq \emptyset$, and $f(X) \subseteq U \cup V$.
	Because $f$ is continuous, we know that $f^{-1}(U)$ and $f^{-1}(V)$ are open in $X$.
	Then
	\[
	X = f^{-1}(U) \cup f^{-1}(V)
	\]
	is union of two nonempty open sets.
	Moreover, we have
	\[
	f^{-1}(U) \cap f^{-1}(V) = f^{-1}(U \cap V) = f^{-1}(f(X) \cap U \cap V) = f^{-1}(\emptyset) = \emptyset,
	\]
	thus $X$ is disconnected, which leads to a contradiction.
	Therefore we conclude that $f(X)$ is connected.
\end{proof}

\section{Separation Axioms}

\begin{definition}\label{De:AxiomsOfSeparation}
	Let $(X,\tau)$ be a topological space.
	It is said to be
	\begin{enumerate}
		\item $T_0$, if for all $x,y \in X$, $x \neq y$ implies there exists an open set $U$ such that exactly one of $x,y$ belongs to $U$.
		\item $T_1$, if for all $x,y \in X$, $x \neq y$ implies there exists an open set $U$ such that $x \in U$ and $y \not\in U$.
		\item $T_2$ or \emph{Hausdorff}, if for all $x,y \in X$, $x \neq y$ implies there exist disjoint open sets $U$ and $V$ such that $x \in U$ and $y \in V$.
		\item \emph{regular}, if for all $x \in X$, for all closed sets $C \subseteq X$, $x \not\in C$ implies there exist disjoint open sets $U$ and $V$ such that $x \in U$ and $C \subseteq V$.
		\item $T_3$, if $(X,\tau)$ is $T_1$ and regular.
		\item \emph{completely regular}, if for all $x \in X$, for all closed sets $C \subseteq X$, $x \not\in C$ implies there exists a continuous function $f : X \to [0,1]$ such that $f(x) = 1$ and $f \equiv 0$ on $C$.
		\item $T_{3\frac{1}{2}}$ or \emph{Tikhonov}, if $(X,\tau)$ is $T_0$ and completely regular.
		\item \emph{normal}, if for all disjoint closed sets $C_1,C_2 \subseteq X$, there exist disjoint open sets $U_1$ and $U_2$ such that $C_1 \subseteq U_1$ and $C_2 \subseteq U_2$.
		\item $T_4$, if $(X,\tau)$ is $T_1$ and normal.
	\end{enumerate}
\end{definition}

\begin{proposition}\label{Prop:T1SingletonClosed}
	If $(X,\tau)$ is $T_1$, then every singleton is closed.
\end{proposition}

\begin{theorem}\label{The:HierarchyOfSeparation}
	For the axioms of separation defined in \defref{AxiomsOfSeparation}, we have
	\begin{equation*}
		T_4 \implies T_3 \implies T_2 \implies T_1 \implies T_0.
	\end{equation*}
\end{theorem}

\begin{example}\label{Exa:NotT0}
	There exist topologies that are not $T_0$.
	For example, let $X$ be a set with at least 2 elements and $\tau = \{ \emptyset, X\}$.
	Then $(X,\tau)$ is not $T_0$.
\end{example}

\begin{example}\label{Exa:T0NotImplyT1}
	In general $T_0$ does not imply $T_1$.
	For example, let $X = \bbR$ and $\tau = \{ (a, +\infty) : a \in \bbR \} \cup \{ \emptyset \}$.
	Then $(X,\tau)$ is $T_0$ but not $T_1$.
\end{example}

\begin{example}\label{Exa:NormalNotImplyRegular}
	In general normal spaces are not necessarily regular spaces.
	The topology in \exref{T0NotImplyT1} is an example.
\end{example}

\begin{example}\label{Exa:RegularNotImplyNormal}
	In general regular spaces are not necessarily normal spaces.
	Let's consider the Sorgenfrey line $(\bbR,\tau)$ where $\tau = \{ A \subseteq \bbR : \Forall{x \in A} \Exists{\epsilon > 0} [x,x+\epsilon) \subseteq A \}$.
	The Sorgenfrey plane $(\bbR^2,S)$ is defined as the product topology $(\bbR,\tau) \times (\bbR,\tau)$.
	Then $S$ is regular but not normal.
\end{example}

\begin{lemma}\label{Lem:RegularIffNeighborContainClosure}
	Let $(X,\tau)$ be a topological space.
	Then $X$ is regular iff for all $x \in X$, for all neighborhoods $U \subseteq X$ of $x$, there exists an open set $V$ such that $x \in V \subseteq \overline{V} \subseteq U$.
\end{lemma}

\begin{theorem}\label{The:ProductOfRegularStillRegular}
	Let $(X,\tau_X)$ and $(Y,\tau_Y)$ be two topological spaces.
	Then $(X,\tau_X)$ and $(Y,\tau_Y)$ are regular iff the product topology $(X \times Y, \tau_{X \times Y})$ is regular.
\end{theorem}

\begin{proposition}\label{Prop:MetricT4}
	Every metric space is $T_4$, i.e., $T_1$ and normal.
\end{proposition}

\section{Compactness}

\begin{definition}\label{De:Covers}
	Let $(X,\tau)$ be a topological space.
	An \emph{open cover} of $X$ is any collection of open sets whose union is $X$.
	A \emph{subcover} is a subset of a cover which is still a cover.
\end{definition}

\begin{definition}\label{De:CompactSpace}
	Let $(X,\tau)$ be a topological space.
	\begin{itemize}
		\item $(X,\tau)$ is said to be \emph{compact} if every open cover of $X$ admits a finite subcover.
		\item $(X,\tau)$ is said to be \emph{sequentially compact} if every sequence $\{x_n\}_{n \in \bbN} \subseteq X$ admits a convergent subsequence.
		\item $E \subseteq X$ is said to be \emph{relatively compact} (precompact) if $\overline{E}$ is compact w.r.t. the induced topology.
	\end{itemize}
\end{definition}

\begin{remark}\label{Rem:WeakerPreserveCompactness}
	If two topologies $\tau_1 \subseteq \tau_2$, then $\tau_2$ is compact implies that $\tau_1$ is compact.
\end{remark}

\begin{remark}\label{Rem:CompactSetFiniteSubcover}
	Let $(X,\tau)$ be a topological space and $K \subseteq X$.
	Then $K$ is compact iff $(K, \tau|_K)$ is compact, i.e., if $K \subseteq \bigcup_{\alpha \in \Lambda} U_\alpha$ with a collection of open sets $U_\alpha$, then there exist a finite $\calI \subseteq \Lambda$ such that $K \subseteq \bigcup_{\alpha \in \calI} U_\alpha$.
\end{remark}

\begin{lemma}\label{Lem:HausdorffSeparatePointAndCompact}
	Let $(X,\tau)$ be a Hausdorff space.
	Suppose $K \subseteq X$ is compact and $x_0 \not\in K$.
	Then there exist disjoint open sets $U$ and $V$ such that $x_0 \in V$ and $K \subseteq U$.
\end{lemma}

\begin{proposition}\label{Prop:SeveralPropertiesOfCompactness}
	Let $(X,\tau)$ be a topological space.
	\begin{enumerate}
		\item If $X$ is compact and $C \subseteq X$ is closed, then $C$ is also compact.
		\item If $X$ is Hausdorff and $K \subseteq X$ is compact, then $K$ is closed.
		\item If $X$ is Hausdorff and compact, then $X$ is regular.
		\item If $X$ is Hausdorff and compact, then $X$ is normal.
	\end{enumerate}
\end{proposition}

\begin{example}\label{Exa:CompactNotImplyClosed}
	In general compactness does not imply closedness.
\end{example}

\begin{example}\label{Exa:ClosureOfCompactNotCompact}
	If $(X,\tau)$ is not Hausdorff, there might exist a compact set $K \subseteq X$ but $\overline{K}$ is not compact.
\end{example}

\begin{proposition}\label{Prop:ClosedIntervalCompact}
	If $a,b \in \bbR$ and $a < b$, then $[a,b]$ is compact.
\end{proposition}

\begin{example}\label{Exa:ClosedBoundedNotCompact}
	In general closed bounded sets are not necessarily compact.
\end{example}

\begin{definition}\label{De:FiniteIntersectionProperty}
	A family $\{E_\alpha\}_{\alpha \in \Lambda}$ of subsets of a set $X$ is said to have the \emph{finite intersection property}, if every finite subfamily has a nonempty intersection.
\end{definition}

\begin{proposition}\label{Prop:CompactIffFiniteIntersectClosedSetsNonemptyIntersection}
	Let $(X,\tau)$ be a topological space.
	Then $X$ is compact iff for every family $\{C_\alpha\}_{\alpha \in \Lambda}$ of closed subsets with the finite intersection property, we have $\bigcap_{\alpha \in \Lambda} C_\alpha \neq \emptyset$.
\end{proposition}

\begin{lemma}\label{Lem:InfiniteSubsetOfCompactHasLimit}
	Let $(X,\tau)$ be a compact space.
	If $E \subseteq X$ is infinite then it has an accumulation point.
\end{lemma}

\begin{proposition}\label{Prop:CopactFstCountThenSeqCompact}
	If $(X,\tau)$ is compact and first countable, then $X$ is sequentially compact.
\end{proposition}

\begin{definition}\label{De:TotallyBoundedness}
	Let $(X,d)$ be a metric space.
	$A \subseteq X$ is said to be \emph{totally bounded} if for all $\epsilon > 0$, there exists $n 	\in \bbN$ and $a_1,\cdots,a_n \in A$ such that $A \subseteq \bigcup_{i=1}^n B(a_i,\epsilon)$.
\end{definition}

\begin{lemma}\label{Lem:MetricSeqCompactIsTotallyBounded}
	Let $(X,d)$ be a metric space.
	If $X$ is sequentially compact, then it is totally bounded, so also bounded.
\end{lemma}

\begin{proposition}\label{Prop:MetricCompactIffSeqCompact}
	Let $(X,d)$ be a metric space.
	Then $X$ is compact iff $X$ is sequentially compact.
\end{proposition}

\begin{proposition}\label{Prop:MetricSeqCompactImplySeparable}
	Let $(X,d)$ be a metric space.
	If $X$ is sequentially compact, then $X$ is separable.
	In particular, $X$ is second countable.
\end{proposition}

\begin{proposition}\label{Prop:MetricSndCountImplyLindelof}
	Let $(X,d)$ be a metric space.
	If $X$ is second countable, then it is Lindel{\"o}f, i.e., every open cover admits a countable subcover.
\end{proposition}

\begin{proposition}\label{Prop:ContFuncPreserveCompact}
	Let $(X,\tau_X)$ and $(Y,\tau_Y)$ be two topological spaces.
	If $K \subseteq X$ is (sequentially) compact, $f : X \to Y$ is continuous, then $f(K)$ is also (sequentially) compact.
\end{proposition}

\begin{proposition}\label{Prop:ContBijectionFromCompactToT2IsHome}
	Let $(X,\tau_X)$ and $(Y,\tau_Y)$ be two topological spaces.
	Suppose $f : X \to Y$ is a bijection, $X$ is compact, $Y$ is $T_2$ and $f$ is continuous.
	Then $f^{-1} : Y \to X$ is continuous, i.e., $f$ is a homeomorphism.
\end{proposition}

\begin{example}\label{Exa:ContBijectionNotHome}
	In general \propref{ContBijectionFromCompactToT2IsHome} does not hold if $X$ is not compact or $Y$ is not $T_2$.
\end{example}

\begin{theorem}\label{The:Weierstrass}
	Let $(X,\tau)$ be a (sequentially) compact space.
	If $f: X\to\bbR$ is (sequentially) continuous, then there exists $x \in X$ such that $f(x) = \min_{x \in X} f(x)$.
\end{theorem}

\begin{theorem}\label{The:ProductOfCompactStillCompact}
	Let $(X,\tau_X)$ and $(Y,\tau_Y)$ be two topological spaces.
	Then $(X,\tau_X)$ and $(Y,\tau_Y)$ are compact iff the product topology $(X \times Y,\tau_{X \times Y})$ is compact. 
\end{theorem}

\section{Compactification}

\begin{definition}\label{De:Compactification}
	Let $(X,\tau_X)$ be a topological space.
	$(Y,\tau_Y)$ is said to be a \emph{compactification} of $(X,\tau_X)$ if $Y$ is compact and $X$ is homeomorphic to a dense subset of $Y$.
\end{definition}

\begin{definition}\label{De:LocallyCompactness}
	A topological space $(X,\tau)$ is said to be \emph{locally compact} iff for all $x \in X$, there exists a neighborhood $U$ of $x$ such that $\overline{U}$ is compact.
\end{definition}

\begin{theorem}\label{The:AlexandroffCompactification}
	Let $(X,\tau)$ be a topological space that is not compact.
	Let $\infty$ denote a point that is not in $X$ and $X_\infty \coloneqq X \cup \{\infty\}$.
	Let $\tau_\infty$ be the collection of all subsets $U \subseteq X_\infty$ such that either (i) $U \in \tau$, or (ii) $\infty \in U$ and $X \setminus U$ is closed and compact.
	Then $(X_\infty,\tau_\infty)$ is compact and $X$ is dense in $X_\infty$.
	Moreover, $(X_\infty,\tau_\infty)$ is Hausdorff iff $(X,\tau)$ is Hausdorff and locally compact.
\end{theorem}

\begin{definition}\label{De:GeneralProductTopology}
	Let $\{(X_\alpha,\tau_\alpha)\}_{\alpha \in \Lambda}$ be a collection of topological spaces.
	Consider
	\[
	\prod_{\alpha \in \Lambda} X_\alpha \coloneqq \set{ f : \Lambda \to \bigcup_{\alpha \in \Lambda} X_\alpha : \Forall{\alpha \in \Lambda} f(\alpha) \in X_\alpha }.
	\]
	For an $f$ in the collection defined above, $f(\alpha)$ is said to be the \emph{$\alpha$-coordinate} of $f$.
	The \emph{project} along the $\beta$-axis for any $\beta \in \Lambda$ is defined as $\pi_\beta(f) \coloneqq f(\beta)$.
	The \emph{product topology} on $\prod_{\alpha \in \Lambda} X_\alpha$ is the smallest topology that renders each projection continuous.
	Then a subbase of open sets would be
	\[
	\calF \coloneqq \set{ \pi_\alpha^{-1}(V_\alpha) : \alpha \in \Lambda \wedge V_\alpha \in \tau_\alpha }
	\]
	and a base would be
	\[
	\beta \coloneqq \set{ \bigcap_{\alpha \in \Lambda_o} \pi_\alpha^{-1}(V_\alpha) : \Lambda_o \subseteq \Lambda~\mathrm{finite} \wedge V_\alpha \in \tau_\alpha  }.
	\]
\end{definition}

\begin{definition}\label{De:BoxTopology}
	Let $\{(X_\alpha,\tau_\alpha)\}_{\alpha \in \Lambda}$ be a collection of topological spaces.
	The \emph{box topology} on $\prod_{\alpha \in \Lambda} X_\alpha$ is the topology generated by $\set{ \prod_{\alpha \in \Lambda} V_\alpha : V_\alpha \in \tau_\alpha }$.
\end{definition}

\begin{lemma}\label{Lem:ProdTopSeqConvergeIffEveryAxisConverge}
	Let $X \coloneqq \prod_{\alpha \in \Lambda} X_\alpha$ with the product topology.
	A sequence $\{x^n\}_{n \in \bbN} \subseteq X$ converges to $x^o \in X$ iff $x^n_\alpha \rightarrow x^o_\alpha$ for all $\alpha \in \Lambda$.
\end{lemma}

\begin{lemma}\label{Lem:ProdTopContFuncIffContEveryAxis}
	Let $X \coloneqq \prod_{\alpha \in \Lambda} X_\alpha$ with the product topology and $(Z,\tau_Z)$ be a topological space.
	A function $f :Z \to X$ is continuous iff $\pi_\alpha \circ f : Z \to X_\alpha$ is continuous for all $\alpha \in \Lambda$.
\end{lemma}

\begin{lemma}\label{Lem:CompactIffCompactWRTSubbase}
	Let $(X,\tau)$ be a topological space and $\calS$ be a subbase of $\tau$.
	Then $X$ is compact iff every cover of $X$ with elements of $\calS$ admits a finite subcover.
\end{lemma}

\bibliographystyle{abbrv}
%\bibliography{db}
\end{document}
