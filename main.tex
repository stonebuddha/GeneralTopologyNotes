\documentclass{techreport}

%\nochangebars

\newcommand{\Omit}[1]{}

\sloppy
\begin{document}
\title{21-651: General Topology}
\author{Di Wang (diw3)}
\maketitle

\section{Sets and Ordering}

\begin{definition}\label{De:Equipotent}
	$2$ sets $A,B$ are said to be \emph{equipotent}, written $A \sim B$, if there exists a bijection between them.
\end{definition}

\begin{remark}\label{Rem:EquipotenceIsEquivalence}
	Equipotence is an equivalence relation.
\end{remark}

\begin{definition}\label{De:Countability}\
	\begin{itemize}
		\item A set is said to be \emph{finite} if it is equipotent to a set $\{1,\cdots,n\} \subseteq \bbN$ for some $n \in \bbN$.
		\item A set is said to be \emph{countable} if it is equipotent to some subset of $\bbN$.
		\item A set is said to be \emph{uncountable} if it is not countable.
	\end{itemize}
\end{definition}

\begin{remark}\label{Rem:RIsUncountable}
	$\bbR$ is uncountable.
\end{remark}
\begin{proof}
	Suppose $\bbR$ is countable.
	Then there exists $S \subseteq \bbN^+$ such that $\bbR \sim S$.
	In other words, there exists a bijection $f : \bbR \to S$.
	Define $g$ to be the restriction of $f$ on $[0,1]$.
	Then $g : [0,1] \to f([0,1])$ is also a bijection.
	Define $S' \coloneqq f([0,1]) \subseteq S \subseteq \bbN^+$.
	Define a sequence $\{x_n\}_{n \in \bbN^+} \subseteq \{0,1\}$ in the following way:
	\begin{itemize}
		\item If $n \in S'$, let $r \coloneqq g^{-1}(n)$.
		Then there exists a sequence $\{ r_n\}_{n \in \bbN^+}$ such that $r = \sum_{n=1}^\infty r_n \cdot 2^{-n}$.
		Define $x_n \coloneqq 1 - r_n$.
		\item If $n \not\in S'$, define $x_n \coloneqq 0$.
	\end{itemize}
	Define $x \coloneqq \sum_{n=1}^\infty x_n \cdot 2^{-n} \in [0,1]$.
	We claim that $g(x) \not\in S'$.
	If $g(x) \in S'$, then $x = g^{-1}(g(x))$. By the definition of $x$ we know that $x_{g(x)} = 1-x_{g(x)}$, which leads to a contradiction.
	Hence $g(x) \not\in S'$, and therefore $g$ is not a bijection.
	Then we can conclude that $\bbR$ is not countable, i.e., indeed uncountable.
\end{proof}

\begin{theorem}[Cantor-Bernstein-Schr{\"o}der]\label{The:CantorBernsteinSchroder}
	If there exist injections $f:A \to B$ and $g : B \to A$, then $A \sim B$.
\end{theorem}

\begin{definition}\label{De:PowerSet}
	Let $X$ be a set.
	The \emph{power set} of $X$, written $\wp(X)$, is the set of all subsets of $X$.
\end{definition}

\begin{lemma}[Cantor]\label{Lem:CantorPowerSet}
	Let $X$ be a set. There is no surjection from $X$ to $\wp(X)$.
\end{lemma}

\begin{remark}\label{Rem:PowerSetAndIndicators}
	Let $X$ be a set. Then
	\begin{equation*}
		\wp(X) \sim \{0,1\}^X \coloneqq \set{ f : X \to \{0,1\} }.
	\end{equation*}
\end{remark}

\begin{definition}\label{De:PartialOrder}
	A \emph{partial order} on a set $X$ is a binary relation on $X$, written ${\le}$, if
	\begin{itemize}
		\item (reflexivity): $x \le x$ for all $x \in X$,
		\item (antisymmetry): $x \le y$ and $y \le x$ imply $x = y$ for all $x,y \in X$, and
		\item (transitivity): $x \le y$ and $y \le z$ imply $x \le z$ for all $x,y,z \in X$.
	\end{itemize}
	Then $(X,{\le})$ is said to be a \emph{partial ordering}, or \emph{poset}.
\end{definition}

\begin{example}\label{Exa:RSubsetIsPoset}
	Let $X \coloneqq \wp(\bbR)$.
	Then $A \le B \coloneqq A \subseteq B$ is a partial order on $X$.
\end{example}

\begin{definition}\label{De:LinearOrder}
	A partial ordering $(X,{\le})$ is a \emph{linear ordering} if for all $x,y \in X$, we have either $x \le y$ or $y \le x$.
\end{definition}

\begin{definition}\label{De:BoundingElementsInPoset}
	Let $(X,{\le})$ be a partial ordering.
	Let $E \subseteq X$.
	\begin{itemize}
		\item $x \in X$ is said to be an \emph{upper bound} of $E$ if for all $y \in E$, we have $y \le x$.
		\item $x \in E$ is said to be a \emph{maximal element} of $E$ if there does not exist $y \in E$ such that $x < y$.
		\item If $x \in E$ and for all $y \in E$ we have $y \le x$, then $x$ is the \emph{greatest element} of $E$.
	\end{itemize}
	\emph{Lower bounds}, \emph{minimal elements}, and \emph{least elements} are defined in a similar way.
\end{definition}

\begin{definition}\label{De:WellOrdering}
	A poset $(X,{\le})$ is said to be \emph{well-ordered} if it is linearly ordered and every nonempty subset of $X$ has the least element.
\end{definition}

\begin{theorem}\label{The:EquivalentZFC}
	The following three are equivalent:
	\begin{itemize}
		\item \emph{Axiom of Choice:} Given a collection $\calA$ of nonempty disjoint sets, there exists a set $C$ such that (i) $C \subseteq \bigcup \calA$, and (ii) for all $A \in \calA$, $C \cap A$ has exactly one element.
		\item \emph{Zorn's Lemma:} Suppose $(X,{\le})$ is a poset. If every chain in $X$ has an upper bound, then $X$ has a maximal element.
		\item \emph{Well Ordering Lemma (Zermelo's Theorem):} Every nonempty set $X$ admits a well ordering ${\le}$.
	\end{itemize}
\end{theorem}

\section{Topological Spaces}

\begin{definition}\label{De:Topology}
	Let $X$ be a nonempty set.
	A collection $\tau \subseteq \wp(X)$ is said to be a \emph{topology} if
	\begin{enumerate}
		\item $\emptyset,X \in \tau$,
		\item $\tau$ is closed under finite intersection, and
		\item $\tau$ is closed under arbitrary union.
	\end{enumerate}
	Then $(X,\tau)$ is said to be a \emph{topological space}.
	Elements of $\tau$ are said to be \emph{open sets}.
\end{definition}

\begin{example}\label{Exa:BasicTopologies}\
		\begin{itemize}
		\item \emph{Trivial Topology:} $\tau \coloneqq \{ \emptyset, X\}$.
		\item \emph{Discrete Topology:} $\tau \coloneqq \wp(X)$.
		\item \emph{Standard Topology in $\bbR$:} $A$ is an open set (i.e., $A \in \tau$) iff for all $x \in A$, there exists $\epsilon > 0$, such that $(x-\epsilon,x+\epsilon) \subseteq A$.
		\item \emph{Sorgenfrey (Line) Topology in $\bbR$:} $A \in \tau$ iff for all $x \in A$, there exists $\epsilon > 0$, such that $[x,x+\epsilon) \subseteq A$.
		\item $\tau_r \coloneqq \{ (a,+\infty) : a \in \bbR \} \cup \{ \emptyset,\bbR \}$ and $\tau_l \coloneqq \{ (-\infty,a) : a \in \bbR \} \cup \{ \emptyset,\bbR\}$ are two topologies in $\bbR$.
	\end{itemize}
\end{example}

\begin{definition}\label{De:TopComparison}
	Let $\tau_1$ and $\tau_2$ be topologies on $X$.
	$\tau_1$ is said to be \emph{coarser than} $\tau_2$ (i.e., $\tau_2$ is said to be \emph{finer than} $\tau_1$) if $\tau_1 \subseteq \tau_2$.
	The trivial topology is the coarsest and the discrete topology is the finest.
\end{definition}

\begin{definition}\label{De:IsolatedPoint}
	$x \in X$ is said to be an \emph{isolated point} if $\{x\}$ is open.
\end{definition}

\begin{remark}\label{Rem:IntersectTopStillTop}
	If $\{ \tau_\alpha \}_{\alpha \in \calI}$ is a collection of topologies on $X$, then
	\begin{equation*}
		\bigcap_{\alpha \in \calI} \tau_\alpha = \set{ U \subseteq X : \Forall{\alpha \in \calI} U \in \tau_\alpha }
	\end{equation*}
	is still a topology on $X$.
\end{remark}

\section{Bases and Subbases}

\begin{proposition}\label{Prop:Subbase}
	Let $X$ be a set and $\calF$ be a family of subsets of $X$.
	Then the smallest topology $\tau$ containing $\calF$ is given by arbitrary union of finite intersection of elements of $\calF \cup \{\emptyset,X\}$, i.e.,
	\begin{equation*}
		\tau = \set{ \bigcup_{\alpha \in \calI} (U_1^\alpha \cap \cdots \cap U_{N(\alpha)}^\alpha) : N(\alpha) \in \bbN, U_i^\alpha \in \calF \cup \{ \emptyset, X \} }.
	\end{equation*}
	Moreover, we say that $\tau$ is \emph{generated} by $\calF$, and $\calF$ is a \emph{subbase} of $\tau$.
\end{proposition}

\begin{definition}\label{De:Neighborhoods}
	Let $(X,\tau)$ be a topological space.
	$U \subseteq X$ is said to be a \emph{neighborhood} of $x \in X$ if $U$ is open and $x \in U$.
	Similarly, $U$ is said to be a neighborhood of $E \subseteq X$ if $U$ is open and $E \subseteq U$.
\end{definition}

\begin{definition}\label{De:BaseLocalBase}
	A family $\beta \subseteq \wp(X)$ is said to be a \emph{base} of the topology $\tau$ on $X$, if every open set can be written as a union of elements of $\beta$.
	
	Given $x \in X$, a family $\beta_x \subseteq \tau$ of neighborhoods of $x$ is said to be a \emph{local base} of $x$ if every neighborhood of $x$ contains an element of $\beta_x$.
\end{definition}

\begin{example}\label{Exa:BaseLocalBase}\
	\begin{itemize}
		\item Consider $(\bbR,\tau_{\mathrm{standard}})$. Then $\beta \coloneqq \{ (a,b) : a,b \in \bbQ, a \le b \}$ is a base.
		\item Consider $(\bbR,\tau_{\mathrm{sorg}})$.
		Then $\beta \coloneqq \{ [x,p) : x \in \bbR, x \le p, p \in \bbQ\}$ is a base.
		For all $x \in \bbR$, $\beta_x \coloneqq \{ [x,p) : x \le p, p \in \bbQ \}$ is a local base of $x$.
		\item Consider $(X,\tau_{\mathrm{discrete}})$. Then $\beta \coloneqq \{ \emptyset \} \cup \{ \{x \} : x \in X \}$ is a base.
	\end{itemize}
\end{example}

\begin{proposition}\label{Prop:BaseStructure}
	Suppose $X \neq \emptyset$ and $\beta \subseteq \wp(X)$.
	Then $\beta$ is a base for a topology on $X$ iff
	\begin{enumerate}
		\item $\emptyset \in \beta$,
		\item for all $x \in X$, there exists $B \in \beta$, such that $x \in B$, and
		\item for all $B_1,B_2 \in \beta$, $B_1 \cap B_2 \neq \emptyset$, then for all $x \in B_1 \cap B_2$, there exist $B_3 \in \beta$, such that $x \in B_3$ and $B_3 \subseteq B_1 \cap B_2$.
	\end{enumerate}
\end{proposition}

\section{Countability Axioms}

\begin{definition}\label{De:AxiomsOfCountability}
	Let $(X,\tau)$ be a topological space.
	\begin{itemize}
		\item $(X,\tau)$ is said to be \emph{first countable} (i.e., $(X,\tau)$ satisfies the first axiom of countability), if every $x \in X$ admits a countable local base.
		\item $(X,\tau)$ is said to be \emph{second countable}, if it has a countable base.
	\end{itemize}
\end{definition}

\begin{remark}\label{Rem:SndAxCountImplyFstAx}
	The second axiom of countability implies the first one, but the other direction does not hold in general.
\end{remark}

\section{Topological Constructions}

\begin{definition}\label{De:SomeMoreTopologies}\
	\begin{itemize}
		\item \emph{Induced Topology:} Let $(X,\tau)$ be a topological space and $Y \subseteq X$.
		Then $\tau_Y \coloneqq \{ Y \cap U : U \in \tau \}$ is a topology on $Y$.
		\item \emph{Inverse Image Topology:} Let $f : X \to Y$ and $(Y,\tau_Y)$ be a topological space. Then $\tau_X \coloneqq \{ f^{-1}(V) : V \in \tau_Y \}$ is a topology on $X$.
		\item \emph{Direct Image Topology:} Let $f : X \to Y$ and $(X,\tau_X)$ be a topological space. Then $\tau_Y \coloneqq \{ V : f^{-1}(V) \in \tau_X \}$ is a topology on $Y$.
		\item \emph{Quotient Topology:} Let $(X,\tau)$ be a topological space and ${\sim}$ be an equivalence relation in $X$.
		For all $x \in X$, define $[x] \coloneqq \{ y \in X : y \sim x \}$ to be the equivalence class of $x$.
		Define $Y \coloneqq X /{\sim}$ and a projection map $p : X \to Y$ as $x \mapsto [x]$.
		The \emph{quotient topology} on $Y$ is the direct image topology of $\tau$ via $p$.
		\item \emph{Product Topology:} Let $(X,\tau_X)$ and $(Y,\tau_Y)$ be two topological spaces.
		The \emph{product topology} on $X \times Y$ is given by a subbase $\calF \coloneqq \{ U \times V : U \in \tau_X, V \in \tau_Y \}$.
	\end{itemize}
\end{definition}

\begin{remark}\label{Rem:DirectThenInverseIsWeaker}
	Let $(X,\tau_X)$ be a topological space and $f : X \to Y$.
	Let $\tau_Y$ be the direct image topology of $\tau_X$ via $f$ and $\tilde{\tau}_X$ be the inverse image topology of $\tau_Y$ via $f$.
	Then $\tilde{\tau}_X \subseteq \tau_X$.

	As an example, consider the topology $(\bbR,\tau_{\mathrm{standard}})$.
	Define $f : \bbR \to \bbR$ as $x \mapsto 0$.
	Then the direct image topology $\tau_\bbR$ is $\{ V \subseteq \bbR : f^{-1}(V) \in \tau_\mathrm{standard} \}$, i.e., the discrete topology $\wp(\bbR)$.
	The inverse image topology is then $\tilde{\tau}_{\bbR} = \{ f^{-1}(V) : V \in \tau_\bbR \}$, i.e., the trivial topology $\{ \emptyset, \bbR \}$.
\end{remark}

\begin{definition}\label{De:HereditaryProperty}
	A property of topological spaces is said to be \emph{hereditary} if whenever a space has the property, so do all its subsets with induced topology.
\end{definition}

\begin{remark}\label{Rem:AxCountHereditary}
	The first/second axiom of countability are hereditary properties.
\end{remark}

\begin{remark}\label{Rem:SomeResultsAboutTopologies}\
	\begin{itemize}
		\item Let $(X,\tau_X)$ be a topological space and $Y \subseteq X$.
		Let $\tau_Y$ be the induced topology on $Y$.
		Then $Y$ is open in $X$ iff $\tau_Y \subseteq \tau_X$.
		\item Let $(X,\tau_X)$ be a topological space and $i : Y \to X$ be an inclusion map (i.e., $y \mapsto y$).
		Then the induced topology on $Y$ is exactly the same as the inverse image topology via $i$.
		\item Let $(X,\tau_X)$ be a topological space and $f : X\to Y$ for some $Y$.
		Let $\tau_Y$ be the direct image topology via $f$ and $E \coloneqq Y \setminus f(X)$.
		Then the induced topology from $Y$ to $E$ is the discrete topology.
		\item Let $(X,\tau_X)$ and $(Y,\tau_Y)$ be two topological spaces.
		Define $\pi_X : X \times Y \to X$ as $(x,y) \mapsto x$ and $\pi_Y : X \times Y \to Y$ as $(x,y) \mapsto y$.
		Let $\sigma_1$ and $\sigma_2$ be the inverse image topology on $X \times Y$ via $\pi_X$ and $\pi_Y$, respectively.
		Let $\tau$ be the topology on $X \times Y$ generated by $\sigma_1 \cup \sigma_2$.
		then $\tau$ is the product topology.
	\end{itemize}	
\end{remark}

\section{Metric Spaces}

\begin{definition}\label{De:Metrics}
	A map $d : X \times X \to [0,+\infty)$ is said to be a \emph{metric} (\emph{distance}), if
	\begin{enumerate}
		\item 
		\begin{enumerate}
			\item $d(x,x) = 0$ for all $x \in X$,
			\item $x \neq y$ implies $d(x,y) > 0$ for all $x,y \in X$,
		\end{enumerate}
		\item $d(x,y) = d(y,x)$ for all $x,y \in X$, and
		\item $d(x,z) \le d(x,y) + d(y,z)$ for all $x,y,z \in X$.
	\end{enumerate}
	Then $(X,d)$ is said to be a \emph{metric space}.
\end{definition}

\begin{definition}\label{De:OpenBalls}
	Let $(X,d)$ be a metric space.
	The \emph{ball} centered at $x \in X$ with radius $r > 0$, written $B(x,r)$, is defined as $\{ y \in X : d(x,y) < r\}$.
\end{definition}

\begin{lemma}\label{Lem:MetricInduceTop}
	$\beta \coloneqq \{ B(x,r) : x \in X, r > 0 \} \cup \{ \emptyset \}$ is a base for a topology.
	In other words, every metric space $(X,d)$ admits a natural topology $\tau_d$.
\end{lemma}

\begin{example}\label{Exa:MetricSpaces}\
	\begin{itemize}
		\item Let $X = \bbR$ and $d(x,y) \coloneqq \abs{x-y}$ be a metric in $X$.
		Then $B(x,\epsilon) = (x-\epsilon,x+\epsilon)$, i.e., the metric space $(X,d)$ admits the standard topology.
		\item Suppose $X$ is a set.
		Define $\tilde{L}_\infty(X) \coloneqq \{ f :  X \to \bbR : f~\text{is bounded} \}$.
		Then $d_\infty(f,g) \coloneqq \sup_{x \in X} \abs{f(x) - g(x)}$ is a metric.
		\item Consider $[a,b] \subseteq \bbR$ and $C([a,b]) = \{ f : [a,b] \to \bbR : f~\text{is continuos}\}$.
		Then $d(f,g) \coloneqq \max_{x \in [a,b]} \abs{f(x)-g(x)}$ is a metric.
		\item Consider the space $\bbR^N$.
		Define
		\begin{align*}
			d_1(x,y) & \coloneqq \sqrt{\sum_{i=1}^N (x_i-y_i)^2} \\
			d_2(x,y) & \coloneqq \sum_{i=1}^N \abs{x_i - y_i} \\
			d_3(x,y) & \coloneqq \max_{i=1,\cdots,N} \abs{x_i - y_i}
		\end{align*}
		then the topologies associated to these metrics coincide.
	\end{itemize}
\end{example}

\begin{remark}\label{Rem:ClosureOfBallsNotClosedBalls}
	Suppose $(X,d)$ is a metric space, $x \in X$ and $r > 0$.
	In general $\overline{B(x,r)} \neq \{y \in X : d(x,y) \le r \}$.
\end{remark}

\begin{definition}\label{De:Diameter}
	Let $(X,d)$ be a metric space and $A \subseteq X$.
	The \emph{diameter} of $A$, written $\mathrm{diam}(A)$, is defined as $\sup \{ d(x,y) : x,y \in A$.
	$A$ is said to be \emph{bounded} if $\mathrm{diam}(A) < +\infty$.
\end{definition}

\begin{proposition}\label{Prop:MetricCanAlwaysBounded}
	Let $(X,d)$ be a metric space.
	Define $d_1(x,y) \coloneqq \frac{d(x,y)}{1 + d(x,y)}$.
	Then $(X,d_1)$ is a bounded metric space and $\tau_d = \tau_{d_1}$.
\end{proposition}

\begin{lemma}\label{Lem:SmallerMetricIsWeaker}
	If $d_1,d_2$ are metrics on $X$ and $d_1 \le d_2$, then $\tau_{d_1} \subseteq \tau_{d_2}$.
\end{lemma}

\begin{remark}\label{Rem:SmallerMetricIndeedWeaker}
	Consider $C((0,1)) = \{ f : (0,1) \to \bbR : f~\text{is continuous} \}$.
	Define $K_n \coloneqq \sq{\frac{1}{n}, 1-\frac{1}{n}}$ for $n \in \bbN^+$.
	Then $\bigcup_{n=1}^\infty K_n = (0,1)$.
	Define
	\begin{align*}
		d(f,g) & \coloneqq \max_{n \in \bbN^+} \set{ \frac{1}{2^n} \max_{x \in K_n} \frac{\abs{f(x)-g(x)}}{1+\abs{f(x)-g(x)}} } \\
		\tilde{d}(f,g) & \coloneqq  \sup_{x \in (0,1)} \frac{\abs{f(x)-g(x)}}{1+\abs{f(x)-g(x)}}
	\end{align*}
	to be two metrics on $C(0,1)$.
	Then $\tau_d \subset \tau_{\tilde{d}}$.
\end{remark}

\begin{definition}\label{De:QuotientMetricSpaces}
	Let $\rho$ be a pseudo-metric (i.e., a metric without the property as \defref{Metrics}(1)(b)) on $X$.
	Define an equivalence relation ${\sim}$ as $x \sim y$ iff $\rho(x,y) = 0$.
	Then $(Y,d)$ with $Y \coloneqq X / {\sim}$ and $d([x],[y]) \coloneqq \rho(x,y)$ is said to be the \emph{quotient metric space}.
\end{definition}

\begin{definition}\label{De:InifiniteSums}
	Let $X$ be a set and $f : X \to [0,+\infty]$.
	The \emph{infinite sum}, written $\sum_{x \in X} f(x)$, is defined as
	\begin{equation*}
		\sup \set{ \sum_{x \in Y} f(x) : Y \subseteq X, Y~\text{is finite} }.
	\end{equation*}
\end{definition}

\begin{proposition}\label{Prop:FiniteSumImplyCountableNonZero}
	If $\sum_{x \in X} f(x) < +\infty$, then the set $\{ x \in X : f(x) > 0\}$ is countable.
	Moreover, $f$ does not take the value $+\infty$.
\end{proposition}

\begin{definition}\label{De:Norms}
	Let $V$ be a vector space over $\bbR$.
	The function $\norm{\cdot} : V \to [0,+\infty)$ is said to be a \emph{norm} on $X$ if
	\begin{enumerate}
		\item $\norm{v} = 0$ iff $v = 0$ for all $v \in V$,
		\item $\norm{\alpha v} = \abs{\alpha} \norm{v}$ for all $v \in V$, $\alpha \in \bbR$, and
		\item $\norm{v+w} \le \norm{v} + \norm{w}$ for all $v,w \in V$.
	\end{enumerate}
	Then $(V,\norm{\cdot})$ is said to be a \emph{normed space}.
\end{definition}

\begin{proposition}\label{Prop:NormInduceMetric}
	If $(X,\norm{\cdot})$ is a normed space, then $d : X \times X \to [0,+\infty)$, defined as $(x,y) \mapsto \norm{x-y}$, is a metric on $X$.
\end{proposition}

\begin{remark}\label{Rem:NormedBalls}
	If $(X,\norm{\cdot})$ is a normed space, we will consider in $X$ the topology induced by $d$ as in \propref{NormInduceMetric}, i.e., the topology whose base of open sets are $\{ B(x,r) : x \in X, r > 0 \}$ with $B(x,r) \coloneqq \{ y \in X : \norm{x-y} < r \}$.
\end{remark}

\begin{remark}\label{Rem:MetricNotImplyNorm}
	In general, topological spaces don't imply metric spaces and metric spaces don't imply normed spaces.
\end{remark}

\begin{definition}\label{De:lpSpaces}
	Given a set $X$ and $p \in [1,+\infty)$, the space $\ell^p(X)$ is defined as
	\begin{equation*}
		\set{ f : X \to \bbR : \sum_{x \in X} \abs{f(x)}^p < +\infty }.
	\end{equation*}
\end{definition}

\begin{definition}[H{\"o}lder's conjugate exponents]\label{De:HoldersConjugateExponent}
	Suppose $1 \le p \le +\infty$.
	The \emph{H{\"o}lder's conjugate exponent} of $p$ is $p' \coloneqq \frac{p}{1-p} \in [1,+\infty]$.
\end{definition}

\begin{theorem}[Yang's inequality]\label{The:YangsInequality}
	Given $1 < p < +\infty$ and $a,b > 0$, we have $ab \le \frac{1}{p} a^p + \frac{1}{p'} b^{p'}$.
\end{theorem}

\begin{theorem}[H{\"o}lder's inequality]\label{The:HoldersInequality}
	Let $X$ be a set.
	Given $1 \le p \le +\infty$, $q$ conjugate exponent of $p$, and $f,g: X \to [-\infty,+\infty]$, we have
	\begin{alignat*}{2}
		\sum_{x \in X} \abs{f(x) \cdot g(x)} & \le  \p{ \sum_{x \in X} \abs{f(x)}^p }^{\frac{1}{p}} \p{  \sum_{y \in X} \abs{g(y)}^q }^{\frac{1}{q}}, \quad &&  1 < p < +\infty \\
		\sum_{x \in X} \abs{f(x) \cdot g(x)} & \le \p{ \sum_{x \in X} \abs{f(x)} } \cdot \sup_{y \in X} \abs{g(y)}, \quad &&  p=1 \\
		\sum_{x \in X} \abs{f(x) \cdot g(x)} & \le \sup_{x \in X} \abs{f(x)} \cdot \p{ \sum_{y \in X} \abs{g(y)} }, \quad && p=+\infty
	\end{alignat*}
	As a corollary, if $f \in \ell^p(X)$ and $g \in \ell^q(X)$, then $f \cdot g \in \ell^1(X)$.
\end{theorem}

\begin{theorem}[Minkowski's inequality]\label{The:MinkowskisInequality}
	Let $X$ be a set.
	Given $1 \le p < +\infty$ and $f,g:X \to [-\infty,+\infty]$, we have
	\begin{equation*}
		\p{ \sum_{x \in X} \abs{f(x)+g(x)}^p }^{\frac{1}{p}} \le \p{ \sum_{x \in X} \abs{f(x)}^p }^{\frac{1}{p}} + \p{ \sum_{x \in X} \abs{g(x)}^p }^{\frac{1}{p}}.
	\end{equation*}
\end{theorem}

\begin{proposition}\label{Prop:lpVectorSpace}
	$\ell^p(X)$ is a vector space over $\bbR$.
\end{proposition}

\section{Limit Points}

\begin{definition}\label{De:LimitPoint}
	Let $(X,\tau)$ be a topological space.
	Given $E \subseteq X$ and $p \in X$, $p$ is said to be a \emph{limit point} of $E$ if for all neighborhoods $U$ of $p$, we have $(U \cap E) \setminus \{ p \} \neq \emptyset$.
\end{definition}

\begin{remark}\label{Rem:LimitContainedInClosure}
	Let $E'$ be the set of all limit points of $E$.
	Then $E' \subseteq \overline{E}$.
\end{remark}

\begin{proposition}\label{Prop:NonLimitIsIsolated}
	Let $(X,\tau)$ be a topological space.
	Suppose $E \subseteq X$ is equipped with the induced topology $\tau_E$.
	Then $p \in E \setminus E'$ iff $\{p\} \in \tau_E$ (i.e., $p$ is an isolated point in $\tau_E$).
\end{proposition}

\begin{proposition}\label{Prop:BallOfLimitContainInfinite}
	Let $(X,d)$ be a metric space.
	Given $E \subseteq X$ and $p \in E'$, for all $r > 0$, the ball $B(p,r)$ contains infinitely many points of $E$.
\end{proposition}

\begin{proposition}\label{Prop:OrigPlusLimitIsClosure}
	Let $(X,\tau)$ be a topological space and $E \subseteq X$.
	Then $E \cup E' = \overline{E}$.
\end{proposition}

\section{Sequences}

\begin{definition}\label{De:Sequences}
	Let $(X,\tau)$ be a topological space.
	Suppose $\{x_n\}_{n \in \bbN} \subseteq X$ is a sequence in $X$.
	Then $p$ is said to be a \emph{limit} of $\{x_n\}_{n \in \bbN}$, written $x_n \xrightarrow[n \to \infty]{X} p$, if for all neighborhoods $U$ of $p$, there exists $n_0 \in \bbN$ such that for all $n \ge n_0$, we have $x_n \in U$.
\end{definition}

\begin{remark}\label{Rem:LimitOfSeqNotUnique}
	The limit of a sequence may not be unique.
\end{remark}

\begin{lemma}\label{Lem:LimitOfSeqInClosure}
	Let $(X,\tau)$ be a topological space.
	Suppose $x_n \rightarrow p$ in $X$ for some $\{x_n\}_{n \in \bbN} \subseteq X$ and $p \in X$.
	If there exists $E \subseteq X$ such that for all $n \in \bbN$ we have $x_n \in E$, then $p \in \overline{E}$.
\end{lemma}

\begin{lemma}\label{Lem:FirstCountLimitOfSeqIffClosure}
	If a topological space satisfies the first axiom of countability, then $p \in \overline{E}$ iff there exists a sequence $\{x_n\}_{n \in \bbN} \subseteq E$ such that $x_n \rightarrow p$.
\end{lemma}

\begin{example}\label{Exa:CountableNotImplyFstCountable}
	In general countable sets do not imply the first axiom of countability.
	Define $X \coloneqq \bbN \times \bbN$.
	Define $\tau$ as follows: $U \subseteq X$ is open if (i) either $(0,0) \not\in U$, or (ii) $(0,0) \in U$ implies that $U$ contains all but a finite number of points in all but a finite number of columns.
	Then $\tau$ is not first countable.
\end{example}

\begin{definition}\label{De:SequentiallyClosed}
	Let $(X,\tau)$ be a topological space.
	Then $E \subseteq X$ is said to be \emph{sequentially closed} if $E = \{ p \in X : \Exists{\{x_n\}_{n \in \bbN} \subseteq E } x_n \rightarrow p \}$.
\end{definition}

\begin{definition}\label{De:SequentiallyOpen}
	Let $(X,\tau)$ be a topological space.
	Then $U \subseteq X$ is said to be \emph{sequentially open} if for all $p \in U$, for all $\{x_n\}_{n \in \bbN} \subseteq X$ with $x_n \rightarrow p$, there exists $n_0 \in \bbN$ such that for all $n \ge n_0$, we have $x_n \in U$.
\end{definition}

\begin{remark}\label{Rem:ClosedOpenImplySeqClosedOpen}\
	\begin{itemize}
		\item If $E$ is closed, then $E$ is sequentially closed.
		\item If $U$ is open, then $U$ is sequentially open.
	\end{itemize}
\end{remark}

\begin{remark}\label{Rem:SeqOpenNotOpen}
	In general, sequentially open sets may not be open.
	Consider an uncountable set $X$ and the co-countable topology $\tau \coloneqq \{ X \setminus F : F~\text{is countable} \} \cup \{ \emptyset \}$.
	Then for all $p \in X$, $\{p\}$ is not open, while all nonempty sets $U \subseteq X$ are sequentially open.
\end{remark}

\begin{remark}\label{Rem:ClosureIsSeqClosureThenFUSpace}
	Define $\overline{E}_{\mathrm{seq}} \coloneqq \{ p \in X : \Exists{\{x_n\}_{n \in \bbN} \subseteq E} x_n \rightarrow p \}$.
	Then in general we have $\overline{E}_{\mathrm{seq}} \subseteq \overline{E}$.
	If $(X,\tau)$ is a topological space where for all $E \subseteq X$ we have $\overline{E} = \overline{E}_{\mathrm{seq}}$, then $(X,\tau)$ is said to be a \emph{Fr{\'e}chet-Urysohn space}.
	Then the first axiom of countability indeed implies F-U.
\end{remark}

\begin{definition}\label{De:SequentialSpace}
	Let $(X,\tau)$ be a topological space.
	If every sequentially closed subset $E \subseteq X$ is closed and every sequentially open subset $U \subseteq X$ is open, then $(X,\tau)$ is said to be a \emph{sequential space}.
\end{definition}
	
\section{Separability}

\begin{definition}\label{De:DenseSet}
	Let $(X,\tau)$ be a topological space.
	$E$ is said to be \emph{dense} if $\overline{E} = X$.
\end{definition}

\begin{definition}\label{De:SeparableSpace}
	A topological space $(X,\tau)$ is said to be \emph{separable} if there exists a countable dense set in $X$.
\end{definition}

\begin{remark}\label{Rem:SubsetSeparable}
	If $\tau_1 \subseteq \tau_2$ and $(X,\tau_2)$ is separable, then $(X,\tau_1)$ is also separable.
\end{remark}

\begin{proposition}\label{Prop:SndAxCountImplySeparable}
	If a topological space $(X,\tau)$ satisfies the second axiom of countability, then $(X,\tau)$ is separable.
\end{proposition}

\begin{proposition}\label{Prop:MetricSeparableImplySndAxCount}
	Suppose $(X,d)$ is a metric space.
	Then $(X,d)$ is separable iff it is second countable.
\end{proposition}

\begin{example}\label{Exa:SeparableSpaces}\
	\begin{itemize}
		\item $\bbR$ with the Sorgenfrey topology is separable. Indeed, we have $\overline{\bbQ} = \bbR$.
		\item Suppose $X \neq \emptyset$ is equipped with the discrete topology.
		Then $X$ is separable iff $X$ is countable.
		\item $\ell^p(\bbN)$ for some $1 \le p < +\infty$ is separable.
		A countable dense set could be $\{  \{ (r_1,\cdots,r_n,0,\cdots) \in \ell^p(\bbN) : n \in \bbN, r_i \in \bbQ \}$.
		\item $C([a,b])$ with the norm $\norm{f}_\infty \coloneqq \max_{x \in [a,b]} \abs{f(x)}$ is separable.
	\end{itemize}
\end{example}

\begin{proposition}\label{Prop:SeparableMetricInduceSeparable}
	Suppose $(X,d)$ is a separable metric space and $A \subseteq X$.
	Then $(A,d|_A)$ is also separable.
\end{proposition}

\section{Connectedness}

\begin{definition}\label{De:ConnectedSpace}
	A topological space $(X,\tau)$ is said to be \emph{connected} if the only open sets that are also closed are $\emptyset$ and $X$.
\end{definition}

\begin{remark}\label{Rem:SplitDisconnectedSpace}
	$(X,\tau)$ said to be is \emph{disconnected} if there exists $A \subseteq X$ such that $A \neq \emptyset$, $A \neq X$ and $A$ is both open and closed.
	In other words, $X = A \cup (X \setminus A)$ is a union of two disjoint open sets.
	In general, $(X,\tau)$ is disconnected if there exist nonempty open sets $U_1$ and $U_2$ such that $U_1 \cap U_2 = \emptyset$ and $X = U_1 \cup U_2$.
\end{remark}

\begin{definition}\label{De:ConnectedSet}
	Let $(X,\tau)$ be a topological space.
	$E \subseteq X$ is said to be \emph{connected} if the induced topology $(E,\tau_E)$ is connected.
\end{definition}

\begin{remark}\label{Rem:DisconnectedSubsetNotLifting}
	Let $(X,\tau)$ be a topological space.
	If $E \subseteq X$ is disconnect, then $E = V_1 \cup V_2$ for some disjoint nonempty open sets $V_1,V_2$ in $E$.
	Then there exist open sets $U_1$ and $U_2$ in $X$ such that $V_1 = E \cap U_1$ and $V_2 = E \cap U_2$.
	However, in general $U_1 \cap U_2 \neq \emptyset$.
\end{remark}

\begin{proposition}\label{Prop:MetricDisconnectedSubsetLifting}
	Suppose $(X,d)$ is a metric space.
	Then $E \subseteq X$ is disconnected iff there exist nonempty disjoint open sets $U$ and $V$ in $X$ such that $E \subseteq U \cup V$.
\end{proposition}

\begin{theorem}\label{The:RConnectedConvex}
	$C \subseteq \bbR$ is connected iff $C$ is convex, i.e., an interval.
\end{theorem}

\begin{proposition}\label{Prop:ConnIffEveryTwoPointsConn}
	A topological space $(X,\tau)$ is connected iff for all $x,y\in X$, there exists $E \subseteq X$ such that $x,y\in E$ and $E$ is connected.
\end{proposition}

\begin{proposition}\label{Prop:UnionConnStillConn}
	Let $(X,\tau)$ be a topological space.
	Let $E \subseteq X$.
	If $E = \bigcup_{\alpha \in \Lambda} E_\alpha$ for each $E_\alpha$ connected, and $\bigcap_{\alpha \in \Lambda} E_\alpha$ is nonempty, then $E$ is connected.
\end{proposition}

\begin{proposition}\label{Prop:ClosureOfConnStillConn}
	Let $(X,\tau)$ be a topological space.
	If $E \subseteq X$ is connected, then $\overline{E}$ is also connected.
\end{proposition}

\begin{definition}\label{De:ContinuousFromInterval}
	Let $(X,\tau)$ be a topological space and $[a,b] \subseteq \bbR$.
	A function $r : [a,b] \to X$ is said to be \emph{continuous} if for all $x_0 \in [a,b]$, for all sequences $\{x_n\}_{n \in \bbN} \subseteq [a,b]$, $x_n \xrightarrow{[a,b]} x_0$ implies $r(x_n) \xrightarrow{X} r(x_0)$.
\end{definition}

\begin{lemma}\label{Lem:ContinuousReverseOpenIsOpen}
	Let $(X,\tau)$ be a topological space and $[a,b] \subseteq \bbR$.
	Let $r : [a,b] \to X$ be a continuous function.
	Then for all $U \in \tau$, $r^{-1}(U)$ is open in $[a,b]$.
\end{lemma}

\begin{remark}\label{Lem:ContinuousReverseClosedIsClosed}
	Let $r : [a,b] \to X$ be a continuous function.
	If $C$ is closed in $X$, then $r^{-1}(C)$ is closed in $[a,b]$.
\end{remark}

\begin{definition}\label{De:PathwiseConnected}
	A topological space $(X,\tau)$ is said to be \emph{pathwise connected} if for all $x,y \in X$, there exists a continuous path $r : [a,b] \to X$ such that $r(a) =x$ and $r(b)=y$.
\end{definition}

\begin{proposition}\label{Prop:PathWiseConnImplyConn}
	If the topological space $(X,\tau)$ is pathwise connected, then it is connected.
\end{proposition}

\begin{proposition}\label{Prop:OpenSetInEuclideanSpaceConn}
	Suppose $U \subseteq \bbR^N$ is an open set.
	Then $U$ is connected iff it is pathwise connected.
\end{proposition}

\begin{definition}\label{De:ConnEquivalenceRelation}
	Let $(X,\tau)$ be a topological space.
	Let's define a relation ${\sim}$ as $x \sim y$ iff there exists $E \subseteq X$ such that $x,y\in E$ and $E$ is connected.
	Then ${\sim}$ is an equivalence relation.
	For all $x \in X$, define the equivalence class of $x$ as $E_x \coloneqq \{ y \in X : y \sim x \}$.
	Then $E_x = \bigcup_{E \subseteq X, x \in E, E~\text{is connected}} E$, i.e., $E_x$ is a (largest) connected component.
	Moreover, we have $E_x = \overline{E_x}$.
\end{definition}

\begin{proposition}\label{Prop:ClosedConnComponentStillClosed}
	Let $(X,\tau)$ be a topological space and $C \subseteq X$ be a closed set.
	The connected components of $C$ are closed in $X$.
\end{proposition}

\begin{definition}\label{De:TotalDisconnectedness}
	A topological space $(X,\tau)$ is said to be \emph{totally disconnected} if every connected component is a singleton.
\end{definition}

\begin{example}\label{De:QInRIsTotallyDisconn}
	$\bbQ$ in $\bbR$ is totally disconnected.
\end{example}

\section{Functions and Continuity}

\begin{definition}\label{De:ContinuousFunctions}
	Let $(X,\tau_X)$ and $(Y,\tau_Y)$ be two topological spaces.
	A function $f : X \to Y$ is said to be \emph{continuous} if for all open sets $V \subseteq Y$ (i.e., $V \in \tau_Y$), we have $f^{-1}(V) \in \tau_X$.
\end{definition}

\begin{proposition}\label{Prop:SomeEquivResultsAboutContFunc}
	Let $(X,\tau_X)$ and $(Y,\tau_Y)$ be two topological spaces.
	Suppose $f : X \to Y$.
	The following are equivalent:
	\begin{enumerate}
		\item $f$ is continuous,
		\item for all $x \in X$, for all neighborhoods $V$ of $f(x)$ in $Y$, there exists a neighborhood $U$ of $x$ in $X$ such that $f(U) \subseteq V$, and
		\item $C$ is a closed set in $Y$ implies $f^{-1}(C)$ is closed in $X$.
	\end{enumerate}
\end{proposition}

\begin{proposition}\label{Prop:ContFuncIffOnSubbase}
	Let $(X,\tau_X)$ and $(Y,\tau_Y)$ be two topological spaces.
	Let $\calS$ be a subbase of open sets in $\tau_Y$.
	Then $f : X \to Y$ is continuous iff for all $V \in \calS$, $f^{-1}(V) \in \tau_X$.
\end{proposition}

\begin{definition}\label{De:SequentiallyContinuous}
	Let $(X,\tau_X)$ and $(Y,\tau_Y)$ be two topological spaces.
	The function $f : X \to Y$ is said to be \emph{sequentially continuous} if for all $x_0 \in X$, for all sequences $\{x_n\}_{n \in \bbN} \subseteq X$, $x_n \xrightarrow{X} x_0$ implies $f(x_0) \xrightarrow{Y} f(x_0)$.
\end{definition}

\begin{remark}\label{Rem:ContImplySeqCont}
	If $f$ is a continuous function, then it is sequentially continuous.
\end{remark}

\begin{example}\label{Exa:SeqContNotCont}
	Let $X = \bbR$ and $\tau_c$ be the co-countable topology $\{ \emptyset \} \cup \{ \bbR \setminus E : E~\text{is countable} \}$.
	Let $Y = X$ equipped with the discrete topology $\tau_d = \wp(\bbR)$.
	Let $f : X \to Y$ be a function defined as $x \mapsto x$.
	Then $f$ is sequentially continuous but not continuous.
\end{example}

\begin{proposition}\label{Prop:FstAxCountContIffSeqCont}
	Let $(X,\tau_X)$ and $(Y,\tau_Y)$ be two topological spaces.
	Let $f : X\to Y$.
	If $X$ satisfies the first axiom of countability, then $f$ is continuous iff $f$ is sequentially continuous.
\end{proposition}

\begin{proposition}\label{Prop:ContCompIsCont}
	The composition of continuous functions is continuous.
\end{proposition}

\begin{proposition}\label{Prop:ContPreserveConn}
	Let $(X,\tau_X)$ and $(Y,\tau_Y)$ be two topological spaces.
	If $f : X \to Y$ is a continuous function, then $X$ is connected implies that $f(X)$ is connected.
\end{proposition}

\section{Separation Axioms}

\begin{definition}\label{De:AxiomsOfSeparation}
	Let $(X,\tau)$ be a topological space.
	It is said to be
	\begin{enumerate}
		\item $T_0$, if for all $x,y \in X$, $x \neq y$ implies there exists an open set $U$ such that exactly one of $x,y$ belongs to $U$.
		\item $T_1$, if for all $x,y \in X$, $x \neq y$ implies there exists an open set $U$ such that $x \in U$ and $y \not\in U$.
		\item $T_2$ or \emph{Hausdorff}, if for all $x,y \in X$, $x \neq y$ implies there exist disjoint open sets $U$ and $V$ such that $x \in U$ and $y \in V$.
		\item \emph{regular}, if for all $x \in X$, for all closed sets $C \subseteq X$, $x \not\in C$ implies there exist disjoint open sets $U$ and $V$ such that $x \in U$ and $C \subseteq V$.
		\item $T_3$, if $(X,\tau)$ is $T_1$ and regular.
		\item \emph{completely regular}, if for all $x \in X$, for all closed sets $C \subseteq X$, $x \not\in C$ implies there exists a continuous function $f : X \to [0,1]$ such that $f(x) = 1$ and $f \equiv 0$ on $C$.
		\item $T_{3\frac{1}{2}}$ or \emph{Tikhonov}, if $(X,\tau)$ is $T_0$ and completely regular.
		\item \emph{normal}, if for all disjoint closed sets $C_1,C_2 \subseteq X$, there exist disjoint open sets $U_1$ and $U_2$ such that $C_1 \subseteq U_1$ and $C_2 \subseteq U_2$.
		\item $T_4$, if $(X,\tau)$ is $T_1$ and normal.
	\end{enumerate}
\end{definition}

\begin{proposition}\label{Prop:T1SingletonClosed}
	If $(X,\tau)$ is $T_1$, then every singleton is closed.
\end{proposition}

\begin{theorem}\label{The:HierarchyOfSeparation}
	Consider axioms of separation.
	We have
	\begin{equation*}
		T_4 \implies T_3 \implies T_2 \implies T_1 \implies T_0.
	\end{equation*}
\end{theorem}

\begin{example}\label{Exa:NotT0}
	There exist topologies that are not $T_0$.
	For example, let $X$ be a set with at least 2 elements and $\tau = \{ \emptyset, X\}$.
	Then $(X,\tau)$ is not $T_0$.
\end{example}

\begin{example}\label{Exa:T0NotImplyT1}
	In general $T_0$ does not imply $T_1$.
	For example, let $X = \bbR$ and $\tau = \{ (a,+\infty) : a \in \bbR \} \cup \{ \emptyset \}$.
	Then $(X,\tau)$ is $T_0$ but not $T_1$.
\end{example}

\begin{example}\label{Exa:NormalNotImplyRegular}
	In general normal spaces are not necessarily regular spaces.
	The topology in \exref{T0NotImplyT1} is an example.
\end{example}

\begin{example}\label{Exa:RegularNotImplyNormal}
	In general regular spaces are not necessarily normal spaces.
	Let's consider the Sorgenfrey line $(\bbR,\tau)$ where $\tau = \{ A \subseteq \bbR : \Forall{x \in A} \Exists{\epsilon > 0} [x,x+\epsilon) \subseteq A \}$.
	The Sorgenfrey plane $(\bbR^2,S)$ is defined as the product topology $(\bbR,\tau) \times (\bbR,\tau)$.
	Then $S$ is regular but not normal.
\end{example}

\begin{lemma}\label{Lem:RegularIffNeighborContainClosure}
	Let $(X,\tau)$ be a topological space.
	Then $X$ is regular iff for all $x \in X$, for all neighborhoods $U \subseteq X$ of $x$, there exists an open set $V$ such that $x \in V \subseteq \overline{V} \subseteq U$.
\end{lemma}

\bibliographystyle{abbrv}
%\bibliography{db}
\end{document}
